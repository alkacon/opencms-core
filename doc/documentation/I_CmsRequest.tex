\begin{PRE}
All Packages  Class Hierarchy  This Package  Previous  Next  Index
\end{PRE}

\htmlHR

\section*{  Interface com.opencms.core.I\_CmsRequest }

\begin{description}
\item public interface {\bf I\_CmsRequest} 
\end{description}

This interface defines a CmsRequest. The CmsRequest is a genereic request
object that is used in the CmsObject provinding methods to read the data
included in the request. Implementations of this interface use an existing
request (e.g. HttpServletRequest) to initialize a CmsRequest. 

\begin{description}
\item {\bf Version:}  

\$Revision: 1.3 $ \$Date: 2002/03/07 15:58:17 $  
\item {\bf Author:}  

Michael Emmerich, Alexander Kandzior 
\end{description}

\htmlHR

\subsection*{  Method Index }

\begin{description}
\item o {\bf getFile}(String)  

Returns the content of an uploaded file.  
\item o {\bf getFileNames}()  

Returns the names of all uploaded files in this request.  
\item o {\bf getOriginalRequest}()  

Returns the original request that was used to create the CmsRequest.  
\item o {\bf getOriginalRequestType}()  

Returns the type of the request that was used to create the CmsRequest.  
\item o {\bf getParameter}(String)  

Returns the value of a named parameter as a String.  
\item o {\bf getParameterNames}()  

Returns all parameter names as an Enumeration of String objects.  
\item o {\bf getParameterValues}(String)  

Returns all parameter values of a parameter key.  
\item o {\bf getRequestedResource}()  

This funtion returns the name of the requested resource.  
\item o {\bf getScheme}()  

\item o {\bf getServerName}()  

Methods to get the data from the original request.  
\item o {\bf getServerPort}()  

\item o {\bf getServletUrl}()  

Gets the part of the Url that describes the current servlet of this
Web-Application.  
\item o {\bf getWebAppUrl}()  

Returns the part of the Url that descibes the Web-Application. 
\end{description}

\subsection*{  Methods }

o {\bf getFile} 

\begin{PRE}
 public abstract byte[] getFile(String name)
\end{PRE}

\begin{description}
\htmlDD Returns the content of an uploaded file. Returns null if no file with
this name has been uploaded with this request. Returns an empty byte[] if a
file without content has been uploaded. 

\begin{description}
\item {\bf Parameters:}  

name - The name of the uploaded file.  
\item {\bf Returns:}  

The selected uploaded file content.  
\end{description}

\end{description}

o {\bf getFileNames} 

\begin{PRE}
 public abstract Enumeration getFileNames()
\end{PRE}

\begin{description}
\htmlDD Returns the names of all uploaded files in this request. Returns an
empty eumeration if no files were included in the request. 

\begin{description}
\item {\bf Returns:}  

An Enumeration of file names.  
\end{description}

\end{description}

o {\bf getOriginalRequest} 

\begin{PRE}
 public abstract Object getOriginalRequest()
\end{PRE}

\begin{description}
\htmlDD Returns the original request that was used to create the CmsRequest. 

\begin{description}
\item {\bf Returns:}  

The original request of the CmsRequest.  
\end{description}

\end{description}

o {\bf getOriginalRequestType} 

\begin{PRE}
 public abstract int getOriginalRequestType()
\end{PRE}

\begin{description}
\htmlDD Returns the type of the request that was used to create the
CmsRequest. The returned int must be one of the constants defined above in
this interface. 

\begin{description}
\item {\bf Returns:}  

The type of the CmsRequest.  
\end{description}

\end{description}

o {\bf getParameter} 

\begin{PRE}
 public abstract String getParameter(String name)
\end{PRE}

\begin{description}
\htmlDD Returns the value of a named parameter as a String. Returns null if
the parameter does not exist or an empty string if the parameter exists but
without a value. 

\begin{description}
\item {\bf Parameters:}  

name - The name of the parameter.  
\item {\bf Returns:}  

s The value of the parameter.  
\end{description}

\end{description}

o {\bf getParameterNames} 

\begin{PRE}
 public abstract Enumeration getParameterNames()
\end{PRE}

\begin{description}
\htmlDD Returns all parameter names as an Enumeration of String objects.
Returns an empty Enumeratrion if no parameters were included in the request. 

\begin{description}
\item {\bf Returns:}  

Enumeration of parameter names.  
\end{description}

\end{description}

o {\bf getParameterValues} 

\begin{PRE}
 public abstract String[] getParameterValues(String key)
\end{PRE}

\begin{description}
\htmlDD Returns all parameter values of a parameter key. 

\begin{description}
\item {\bf Returns:}  

Aarray of String containing the parameter values.  
\end{description}

\end{description}

o {\bf getRequestedResource} 

\begin{PRE}
 public abstract String getRequestedResource()
\end{PRE}

\begin{description}
\htmlDD This funtion returns the name of the requested resource. 

For a http request, the name of the resource is extracted as follows: {\tt
http://\{servername\}/\{servletpath\}/\{path to the cms resource\}} In the
following example: {\tt
http://my.work.server/servlet/opencms/system/def/explorer} the requested
resource is {\tt /system/def/explorer}. 

\begin{description}
\item {\bf Returns:}  

The path to the requested resource.  
\end{description}

\end{description}

o {\bf getWebAppUrl} 

\begin{PRE}
 public abstract String getWebAppUrl()
\end{PRE}

\begin{description}
\htmlDD Returns the part of the Url that descibes the Web-Application. E.g:
http://www.myserver.com/opencms/engine/index.html returns
http://www.myserver.com/opencms 

\end{description}

o {\bf getServletUrl} 

\begin{PRE}
 public abstract String getServletUrl()
\end{PRE}

\begin{description}
\htmlDD Gets the part of the Url that describes the current servlet of this
Web-Application. 

\end{description}

o {\bf getServerName} 

\begin{PRE}
 public abstract String getServerName()
\end{PRE}

\begin{description}
\htmlDD Methods to get the data from the original request. 

\end{description}

o {\bf getScheme} 

\begin{PRE}
 public abstract String getScheme()
\end{PRE}

o {\bf getServerPort} 

\begin{PRE}
 public abstract int getServerPort()
\end{PRE}

\htmlHR

\begin{PRE}
All Packages  Class Hierarchy  This Package  Previous  Next  Index
\end{PRE}

