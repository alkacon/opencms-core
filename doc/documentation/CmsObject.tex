\begin{PRE}
All Packages  Class Hierarchy  This Package  Previous  Next  Index
\end{PRE}

\htmlHR

\section*{  Class com.opencms.file.CmsObject }

\begin{PRE}
com.opencms.file.CmsObject
\end{PRE}

\htmlHR

\begin{description}
\item public class {\bf CmsObject}  
\item implements I\_CmsConstants 
\end{description}

This class provides access to the OpenCms and its resources. \htmlBR
The CmsObject encapsulates user identification and client requests and is the
central object to transport information in the Cms Servlet. \htmlBR
All operations on the CmsObject are forwarded to the class which extends
A\_CmsRessourceBroker to ensure user authentification in all operations. 

\begin{description}
\item {\bf Version:}  

\$Revision: 1.3 $ \$Date: 2002/03/07 15:58:17 $  
\item {\bf Author:}  

Andreas Schouten, Michaela Schleich, Michael Emmerich 
\end{description}

\htmlHR

\subsection*{  Variable Index }

\begin{description}
\item o {\bf c\_versionNumber}  

The current version-number of OpenCms  
\item o {\bf m\_context}  

The resource broker to access the cms.  
\item o {\bf m\_launcherManager}  

The launcher manager used with this object, Is needed to clear the template
caches.  
\item o {\bf m\_linkSubstitution}  

The class for processing links.  
\item o {\bf m\_mode}  

the modus the cmsObject runs in (used i.e.  
\item o {\bf m\_rb}  

The resource broker to access the cms.  
\item o {\bf m\_sessionStorage}  

Method that can be invoked to find out all currently logged in users. 
\end{description}

\subsection*{  Constructor Index }

\begin{description}
\item o {\bf CmsObject}()  

The default constructor. 
\end{description}

\subsection*{  Method Index }

\begin{description}
\item o {\bf acceptTask}(int)  

Accept a task from the Cms.  
\item o {\bf accessCreate}(String)  

Checks, if the user may create this resource.  
\item o {\bf accessLock}(String)  

Checks, if the user may lock this resource.  
\item o {\bf accessProject}(int)  

Checks if the user can access the project.  
\item o {\bf accessRead}(String)  

Checks, if the user may read this resource.  
\item o {\bf accessWrite}(String)  

Checks, if the user may write this resource.  
\item o {\bf addFileExtension}(String, String)  

Adds a file extension to the list of known file extensions.  
\item o {\bf addGroup}(String, String, int, String)  

Adds a new group to the Cms.  
\item o {\bf addImportUser}(String, String, String, String, String, String,
String, int, Hashtable, String, String, String, int)  

Adds a user to the Cms by import.  
\item o {\bf addUser}(String, String, String, String, Hashtable, int)  

Adds a user to the Cms.  
\item o {\bf addUserToGroup}(String, String)  

Adds a user to a group.  
\item o {\bf addWebUser}(String, String, String, String, Hashtable, int)  

Adds a web user to the Cms.  
\item o {\bf addWebUser}(String, String, String, String, String, Hashtable,
int)  

Adds a web user to the Cms.  
\item o {\bf anonymousUser}()  

Returns the anonymous user object.  
\item o {\bf backupProject}(int, int, long)  

Creates a backup of the published project  
\item o {\bf changeLockedInProject}(int, String)  

Changes the project-id of a resource to the new project for publishing the
resource directly  
\item o {\bf chgrp}(String, String)  

Changes the group of a resource.  
\item o {\bf chgrp}(String, String, boolean)  

Changes the group of a resource.  
\item o {\bf chmod}(String, int)  

Changes the flags of a resource.  
\item o {\bf chmod}(String, int, boolean)  

Changes the flags of a resource.  
\item o {\bf chown}(String, String)  

Changes the owner of a resource.  
\item o {\bf chown}(String, String, boolean)  

Changes the owner of a resource.  
\item o {\bf chtype}(String, String)  

Changes the resourcetype of a resource.  
\item o {\bf clearcache}()  

Clears all internal DB-Caches.  
\item o {\bf copyFile}(String, String)  

Copies a file. {\bf Deprecated.}  
\item o {\bf copyFolder}(String, String)  

Copies a folder. {\bf Deprecated.}  
\item o {\bf copyResource}(String, String)  

Copies a file.  
\item o {\bf copyResource}(String, String, boolean)  

Copies a file.  
\item o {\bf copyResourceToProject}(String)  

Copies a resource from the online project to a new, specified project.  
\item o {\bf copyright}()  

Returns the copyright information for this OpenCms.  
\item o {\bf countLockedResources}(int)  

Counts the locked resources in a project.  
\item o {\bf createChannel}(String, String)  

Creates a new channel.  
\item o {\bf createFile}(String, String, byte[], String)  

Creates a new file with the given content and resourcetype.\htmlBR
{\bf Deprecated.}  
\item o {\bf createFile}(String, String, byte[], String, Hashtable)  

Creates a new file with the given content and resourcetype. {\bf Deprecated.} 

\item o {\bf createFolder}(String, String)  

Creates a new folder. {\bf Deprecated.}  
\item o {\bf createFolder}(String, String, Hashtable)  

Creates a new folder. {\bf Deprecated.}  
\item o {\bf createProject}(String, int, String, long, int)  

Creates a new project for task handling.  
\item o {\bf createProject}(String, String, String, String)  

Creates a new project.  
\item o {\bf createProject}(String, String, String, String, int)  

Creates a new project.  
\item o {\bf createPropertydefinition}(String, String)  

Creates the property-definition for a resource type.  
\item o {\bf createPropertydefinition}(String, String, int)  

Creates the property-definition for a resource type. {\bf Deprecated.}  
\item o {\bf createResource}(String, String, String)  

\item o {\bf createResource}(String, String, String, Hashtable)  

\item o {\bf createResource}(String, String, String, Hashtable, byte[])  

\item o {\bf createTask}(int, String, String, String, String, int, long, int) 


Creates a new task.  
\item o {\bf createTask}(String, String, String, String, long, int)  

Creates a new task.  
\item o {\bf createTempfileProject}()  

Creates a new project for the temporary files.  
\item o {\bf deleteAllProperties}(String)  

Deletes all properties for a file or folder.  
\item o {\bf deleteEmptyFolder}(String)  

Deletes a folder.  
\item o {\bf deleteExportLink}(CmsExportLink)  

Deletes an exportlink in the database.  
\item o {\bf deleteExportLink}(String)  

Deletes an exportlink in the database.  
\item o {\bf deleteFile}(String)  

Deletes a file. {\bf Deprecated.}  
\item o {\bf deleteFolder}(String)  

Deletes a folder. {\bf Deprecated.}  
\item o {\bf deleteGroup}(String)  

Deletes a group.  
\item o {\bf deleteProject}(int)  

Deletes a project.  
\item o {\bf deleteProperty}(String, String)  

Deletes a property for a file or folder.  
\item o {\bf deletePropertydefinition}(String, String)  

Deletes the property-definition for a resource type.  
\item o {\bf deleteResource}(String)  

Deletes a resource.  
\item o {\bf deleteUser}(int)  

Deletes a user from the Cms.  
\item o {\bf deleteUser}(String)  

Deletes a user from the Cms.  
\item o {\bf deleteWebUser}(int)  

Deletes a web user from the Cms.  
\item o {\bf destroy}()  

Destroys the resource borker and required modules and connections.  
\item o {\bf digest}(String)  

Method to encrypt the passwords.  
\item o {\bf doChangeLockedInProject}(int, String)  

Changes the project-id of a resource to the new project for publishing the
resource directly  
\item o {\bf doChgrp}(String, String)  

Changes the group of a resource.  
\item o {\bf doChmod}(String, int)  

Changes the flags of a resource.  
\item o {\bf doChown}(String, String)  

Changes the owner of a resource.  
\item o {\bf doChtype}(String, String)  

Changes the resourcetype of a resource.  
\item o {\bf doCopyFile}(String, String)  

Copies a file.  
\item o {\bf doCopyFolder}(String, String)  

Copies a folder.  
\item o {\bf doCopyResourceToProject}(String)  

Copies a resource from the online project to a new, specified project.  
\item o {\bf doCreateFile}(String, String, byte[], String)  

Creates a new file with the given content and resourcetype.\htmlBR
\item o {\bf doCreateFile}(String, String, byte[], String, Hashtable)  

Creates a new file with the given content and resourcetype.  
\item o {\bf doCreateFolder}(String, String)  

Creates a new folder.  
\item o {\bf doCreateFolder}(String, String, Hashtable)  

Creates a new folder.  
\item o {\bf doDeleteFile}(String)  

Deletes a file.  
\item o {\bf doDeleteFolder}(String)  

Deletes a folder.  
\item o {\bf doLockResource}(String, boolean)  

Locks a given resource.  
\item o {\bf doMoveFile}(String, String)  

Moves a file to the given destination.  
\item o {\bf doRenameFile}(String, String)  

Renames the resource to the new name.  
\item o {\bf doRestoreResource}(int, String)  

Restores a file in the current project with a version in the backup  
\item o {\bf doUndeleteFile}(String)  

Undeletes a file.  
\item o {\bf doUndeleteFolder}(String)  

Undeletes a folder.  
\item o {\bf doUndoChanges}(String)  

Undo changes in a file.  
\item o {\bf doUnlockResource}(String)  

Unlocks a resource.  
\item o {\bf endTask}(int)  

Ends a task of the Cms.  
\item o {\bf exportModuledata}(String, String[], String[])  

Exports channels and moduledata to zip.  
\item o {\bf exportResource}(CmsFile)  

Exports a resource.  
\item o {\bf exportResources}(String, String[])  

Exports cms-resources to a zip-file.  
\item o {\bf exportResources}(String, String[], boolean, boolean)  

Exports cms-resources to a zip-file.  
\item o {\bf exportResources}(String, String[], boolean, boolean, boolean)  

Exports cms-resources to a zip-file.  
\item o {\bf exportStaticResources}(Vector, Vector, CmsPublishedResources)  

Creates a static export in the filesystem  
\item o {\bf forwardTask}(int, String, String)  

Forwards a task to a new user.  
\item o {\bf getAllAccessibleProjects}()  

Returns all projects, which the current user can access.  
\item o {\bf getAllBackupProjects}()  

Returns a Vector with all projects from history  
\item o {\bf getAllManageableProjects}()  

Returns all projects which are owned by the current user or which are
manageable for the group of the user.  
\item o {\bf getAllResourceTypes}()  

Returns a Hashtable with all I\_CmsResourceTypes.  
\item o {\bf getBackupVersionId}()  

Get the next version id for the published backup resources  
\item o {\bf getCacheInfo}()  

Gets information about the cache size.  
\item o {\bf getChild}(String)  

Returns all child groups of a group.  
\item o {\bf getChilds}(String)  

Returns all child groups of a group.  
\item o {\bf getCmsObjectForStaticExport}(CmsExportRequest, CmsExportResponse)
 

Creates a special CmsObject for the static export.  
\item o {\bf getConfigurations}()  

Gets the configurations of the properties-file.  
\item o {\bf getDependingExportLinks}(Vector)  

Reads all export links that depend on the resource.  
\item o {\bf getDirectGroupsOfUser}(String)  

Gets all groups to which a given user directly belongs.  
\item o {\bf getFilesInFolder}(String)  

Returns a Vector with all files of a given folder.  
\item o {\bf getFilesInFolder}(String, boolean)  

Returns a Vector with all files of a given folder.  
\item o {\bf getFilesWithProperty}(String, String)  

Returns a Vector with all resource-names of the resources that have set the
given property to the given value.  
\item o {\bf getFileSystemChanges}()  

This method can be called, to determine if the file-system was changed in the
past.  
\item o {\bf getFileSystemFolderChanges}()  

This method can be called, to determine if the file-system was changed in the
past.  
\item o {\bf getFolderTree}()  

Returns a Vector with the complete folder-tree for this project.\htmlBR
Subfolders can be read from an offline project and the online project.  
\item o {\bf getGroups}()  

Returns all groups in the Cms.  
\item o {\bf getGroupsOfUser}(String)  

Gets all groups of a user.  
\item o {\bf getLauncherManager}()  

Get the launcher manager used with this instance of CmsObject.  
\item o {\bf getLinkRules}(int)  

Returns the ruleset for link replacement.  
\item o {\bf getLinkSubstitution}(String)  

Replaces the link according to the rules and registers it to the requestcontex
if we are in export modus.  
\item o {\bf getLoggedInUsers}()  

Returns a list of all currently logged in users.  
\item o {\bf getMode}()  

Returns the mode this cmsObject is runnig in.  
\item o {\bf getOnlineElementCache}()  

Gets the ElementCache used for the online project.  
\item o {\bf getParent}(String)  

Returns the parent group of a group.  
\item o {\bf getReadingpermittedGroup}(int, String)  

Checks which Group can read the resource and all the parent folders.  
\item o {\bf getRegistry}()  

Gets the Registry.  
\item o {\bf getRequestContext}()  

Returns the current request-context.  
\item o {\bf getResourcesInFolder}(String)  

Returns a Vector with the subresources for a folder.\htmlBR
\item o {\bf getResourcesWithProperty}(String)  

Returns a Vector with all resources of the given type that have set the given
property.  
\item o {\bf getResourcesWithProperty}(String, String, int)  

Returns a Vector with all resources of the given type that have set the given
property to the given value.  
\item o {\bf getResourceType}(int)  

Returns a I\_CmsResourceType.  
\item o {\bf getResourceType}(String)  

Returns a I\_CmsResourceType.  
\item o {\bf getSiteName}()  

Returns the name of the current site, e.g.  
\item o {\bf getSiteRoot}(String)  

Returns the name of the current site root, e.g.  
\item o {\bf getStaticExportPath}()  

Returns the exportpath for the static export.  
\item o {\bf getStaticExportStartPoints}()  

Returns a Vector (of Strings) with the names of the vfs resources (files and
folders) where the export should start.  
\item o {\bf getSubFolders}(String)  

Returns a Vector with all subfolders of a given folder.  
\item o {\bf getSubFolders}(String, boolean)  

Returns a Vector with all subfolders of a given folder.  
\item o {\bf getTaskPar}(int, String)  

Get a parameter value for a task.  
\item o {\bf getTaskType}(String)  

Get the template task id fo a given taskname.  
\item o {\bf getUrlPrefixArray}()  

Gets the prefix array for the linkreplacement  
\item o {\bf getUsers}()  

Returns all users in the Cms.  
\item o {\bf getUsers}(int)  

Returns all users of the given type in the Cms.  
\item o {\bf getUsers}(int, String)  

Returns all users from a given type that start with a specified string  
\item o {\bf getUsersByLastname}(String, int, int, int, int)  

Gets all users with a certain Lastname.  
\item o {\bf getUsersOfGroup}(String)  

Gets all users of a group.  
\item o {\bf getVariantDependencies}()  

Gets the hashtable with the variant dependencies used for the elementcache.  
\item o {\bf importFolder}(String, String)  

Imports a import-resource (folder or zipfile) to the cms.  
\item o {\bf importResource}(String, String, String, String, String, String,
Hashtable, String, byte[], String)  

Imports a resource to the cms.  
\item o {\bf importResources}(String, String)  

Imports a import-resource (folder or zip-file) to the cms.  
\item o {\bf init}(I\_CmsResourceBroker)  

Initializes the CmsObject without a request-context (current-user,
current-group, current-project).  
\item o {\bf init}(I\_CmsResourceBroker, I\_CmsRequest, I\_CmsResponse,
String, String, int, boolean, CmsElementCache, CmsCoreSession)  

Initializes the CmsObject for each request.  
\item o {\bf isAdmin}()  

Checks, if the users current group is the admin-group.  
\item o {\bf isHistoryEnabled}()  

Check if the history is enabled  
\item o {\bf isManagerOfProject}()  

Checks, if the user has management access to the project.  
\item o {\bf isStaticExportEnabled}()  

Returns true, if the static export is enabled.  
\item o {\bf lockedBy}(CmsResource)  

Returns the user, who has locked a given resource.  
\item o {\bf lockedBy}(String)  

Returns the user, who has locked a given resource.  
\item o {\bf lockResource}(String)  

Locks the given resource.  
\item o {\bf lockResource}(String, boolean)  

Locks a given resource.  
\item o {\bf loginUser}(String, String)  

Logs a user into the Cms, if the password is correct.  
\item o {\bf loginWebUser}(String, String)  

Logs a web user into the Cms, if the password is correct.  
\item o {\bf moveFile}(String, String)  

Moves a file to the given destination. {\bf Deprecated.}  
\item o {\bf moveResource}(String, String)  

Moves a resource to the given destination.  
\item o {\bf onlineProject}()  

Returns the online project.  
\item o {\bf publishProject}(int)  

Publishes a project.  
\item o {\bf publishResource}(String)  

Publishes a single resource.  
\item o {\bf readAgent}(CmsTask)  

Reads the agent of a task from the OpenCms.  
\item o {\bf readAllFileHeaders}(String)  

Reads all file headers of a file in the OpenCms. {\bf Deprecated.}  
\item o {\bf readAllFileHeadersForHist}(String)  

Reads all file headers of a file in the OpenCms.  
\item o {\bf readAllProjectResources}(int)  

select all projectResources from an given project  
\item o {\bf readAllProperties}(String)  

Returns a list of all properties of a file or folder.  
\item o {\bf readAllPropertydefinitions}(int)  

Reads all property-definitions for the given resource type.  
\item o {\bf readAllPropertydefinitions}(int, int)  

Reads all property-definitions for the given resource type. {\bf Deprecated.} 

\item o {\bf readAllPropertydefinitions}(String)  

Reads all property-definitions for the given resource type.  
\item o {\bf readAllPropertydefinitions}(String, int)  

Reads all property-definitions for the given resource type. {\bf Deprecated.} 

\item o {\bf readBackupProject}(int)  

Reads a project from the Cms.  
\item o {\bf readCronTable}()  

Gets the Crontable.  
\item o {\bf readExportLink}(String)  

Reads a exportrequest from the Cms.  
\item o {\bf readExportLinkHeader}(String)  

Reads a exportrequest without the dependencies from the Cms.  
\item o {\bf readExportPath}()  

Reads the export-path of the system.  
\item o {\bf readFile}(String)  

Reads a file from the Cms.  
\item o {\bf readFile}(String, boolean)  

Reads a file from the Cms.  
\item o {\bf readFile}(String, String)  

Reads a file from the Cms.  
\item o {\bf readFileExtensions}()  

Gets the known file extensions (=suffixes).  
\item o {\bf readFileForHist}(String, int)  

Reads a file from the Cms for history.  
\item o {\bf readFileHeader}(String)  

Reads a file header from the Cms.  
\item o {\bf readFileHeader}(String, int)  

Reads a file header from the Cms.  
\item o {\bf readFileHeader}(String, String)  

Reads a file header from the Cms.  
\item o {\bf readFileHeaderForHist}(String, int)  

Reads a file header from the Cms for history.  
\item o {\bf readFileHeaders}(int)  

Reads all file headers of a project from the Cms.  
\item o {\bf readFolder}(int, boolean)  

Reads a folder from the Cms.  
\item o {\bf readFolder}(String)  

Reads a folder from the Cms.  
\item o {\bf readFolder}(String, boolean)  

Reads a folder from the Cms.  
\item o {\bf readFolder}(String, String)  

Reads a folder from the Cms.  
\item o {\bf readGivenTasks}(int, String, int, String, String)  

Reads all given tasks from a user for a project.  
\item o {\bf readGroup}(CmsProject)  

Reads the group of a project from the OpenCms.  
\item o {\bf readGroup}(CmsResource)  

Reads the group of a resource from the Cms.  
\item o {\bf readGroup}(CmsTask)  

Reads the group (role) of a task from the Cms.  
\item o {\bf readGroup}(int)  

Reads a group of the Cms.  
\item o {\bf readGroup}(String)  

Reads a group of the Cms.  
\item o {\bf readManagerGroup}(CmsProject)  

Reads the managergroup of a project from the Cms.  
\item o {\bf readMimeTypes}()  

Gets all Mime-Types known by the system.  
\item o {\bf readOriginalAgent}(CmsTask)  

Reads the original agent of a task from the Cms.  
\item o {\bf readOwner}(CmsProject)  

Reads the owner of a project from the Cms.  
\item o {\bf readOwner}(CmsResource)  

Reads the owner of a resource from the Cms.  
\item o {\bf readOwner}(CmsTask)  

Reads the owner (initiator) of a task from the Cms.  
\item o {\bf readOwner}(CmsTaskLog)  

Reads the owner of a tasklog from the Cms.  
\item o {\bf readProject}(CmsResource)  

Reads a project from the Cms.  
\item o {\bf readProject}(CmsTask)  

Reads a project from the Cms.  
\item o {\bf readProject}(int)  

Reads a project from the Cms.  
\item o {\bf readProjectLogs}(int)  

Reads log entries for a project.  
\item o {\bf readProjectView}(int, String)  

Reads all file headers of a project from the Cms.  
\item o {\bf readProperty}(String, String)  

Returns a Property of a file or folder.  
\item o {\bf readPropertydefinition}(String, String)  

Reads the property-definition for the resource type.  
\item o {\bf readTask}(int)  

Reads the task with the given id.  
\item o {\bf readTaskLogs}(int)  

Reads log entries for a task.  
\item o {\bf readTasksForProject}(int, int, String, String)  

Reads all tasks for a project.  
\item o {\bf readTasksForRole}(int, String, int, String, String)  

Reads all tasks for a role in a project.  
\item o {\bf readTasksForUser}(int, String, int, String, String)  

Reads all tasks for a user in a project.  
\item o {\bf readUser}(int)  

Returns a user in the Cms.  
\item o {\bf readUser}(String)  

Returns a user in the Cms.  
\item o {\bf readUser}(String, int)  

Returns a user in the Cms.  
\item o {\bf readUser}(String, String)  

Returns a user in the Cms, if the password is correct.  
\item o {\bf readWebUser}(String)  

Returns a user object if the password for the user is correct. {\bf Security:}
All users are granted.  
\item o {\bf readWebUser}(String, String)  

Returns a user object if the password for the user is correct. {\bf Security:}
All users are granted.  
\item o {\bf reaktivateTask}(int)  

Reactivates a task from the Cms.  
\item o {\bf recoverPassword}(String, String, String)  

Sets a new password if the user knows his recovery-password.  
\item o {\bf removeUserFromGroup}(String, String)  

Removes a user from a group.  
\item o {\bf renameFile}(String, String)  

Renames the file to the new name. {\bf Deprecated.}  
\item o {\bf renameResource}(String, String)  

Renames the resource to the new name.  
\item o {\bf restoreResource}(int, String)  

Restores a file in the current project with a version in the backup  
\item o {\bf rootFolder}()  

Returns the root-folder object.  
\item o {\bf sendBroadcastMessage}(String)  

Returns a list of all currently logged in users.  
\item o {\bf setContextTo}(String)  

Sets the name of the current site root  
\item o {\bf setContextToCos}()  

Sets the name of the current site root of the content objects system  
\item o {\bf setContextToVfs}()  

Sets the name of the current site root of the virtual file system  
\item o {\bf setLauncherManager}(CmsLauncherManager)  

Set the launcher manager used with this instance of CmsObject.  
\item o {\bf setMode}(int)  

Sets the mode this CmsObject runs in.  
\item o {\bf setName}(int, String)  

Set a new name for a task.  
\item o {\bf setParentGroup}(String, String)  

Sets a new parent-group for an already existing group in the Cms.  
\item o {\bf setPassword}(String, String)  

Sets the password for a user.  
\item o {\bf setPassword}(String, String, String)  

Sets the password for a user.  
\item o {\bf setPriority}(int, int)  

Sets the priority of a task.  
\item o {\bf setRecoveryPassword}(String, String, String)  

Sets the recovery password for a user.  
\item o {\bf setTaskPar}(int, String, String)  

Set a parameter for a task.  
\item o {\bf setTimeout}(int, long)  

Sets the timeout of a task.  
\item o {\bf syncFolder}(String)  

Synchronize cms-resources on virtual filesystem with the server filesystem.  
\item o {\bf undeleteResource}(String)  

Undeletes a resource.  
\item o {\bf undoChanges}(String)  

Undo changes in a file by copying the online file.  
\item o {\bf unlockProject}(int)  

Unlocks all resources of a project.  
\item o {\bf unlockResource}(String)  

Unlocks a resource.  
\item o {\bf userInGroup}(String, String)  

Tests, if a user is member of the given group.  
\item o {\bf version}()  

Returns a String containing version information for this OpenCms.  
\item o {\bf writeCronTable}(String)  

Writes the Crontable.  
\item o {\bf writeExportLink}(CmsExportLink)  

Writes an exportlink to the Cms.  
\item o {\bf writeExportLinkProcessedState}(CmsExportLink)  

Sets one exportLink to procecced.  
\item o {\bf writeExportPath}(String)  

Writes the export-path for the system.  
\item o {\bf writeFile}(CmsFile)  

Writes a file to the Cms.  
\item o {\bf writeFileExtensions}(Hashtable)  

Writes the file extensions.  
\item o {\bf writeFileHeader}(CmsFile)  

Writes a file-header to the Cms.  
\item o {\bf writeGroup}(CmsGroup)  

Writes an already existing group to the Cms.  
\item o {\bf writeProperties}(String, Hashtable)  

Writes a couple of Properties for a file or folder.  
\item o {\bf writeProperty}(String, String, String)  

Writes a property for a file or folder.  
\item o {\bf writePropertydefinition}(CmsPropertydefinition)  

Writes the property-definition for the resource type. {\bf Deprecated.}  
\item o {\bf writeTaskLog}(int, String)  

Writes a new user tasklog for a task.  
\item o {\bf writeTaskLog}(int, String, int)  

Writes a new user tasklog for a task.  
\item o {\bf writeUser}(CmsUser)  

Updates the user information.  
\item o {\bf writeWebUser}(CmsUser)  

Updates the user information of a web user. 
\end{description}

\subsection*{  Variables }

o {\bf m\_sessionStorage} 

\begin{PRE}
 private CmsCoreSession m\_sessionStorage
\end{PRE}

\begin{description}
\htmlDD Method that can be invoked to find out all currently logged in users.

\end{description}

o {\bf c\_versionNumber} 

\begin{PRE}
 private static String c\_versionNumber
\end{PRE}

\begin{description}
\htmlDD The current version-number of OpenCms

\end{description}

o {\bf m\_rb} 

\begin{PRE}
 private I\_CmsResourceBroker m\_rb
\end{PRE}

\begin{description}
\htmlDD The resource broker to access the cms.

\end{description}

o {\bf m\_context} 

\begin{PRE}
 private CmsRequestContext m\_context
\end{PRE}

\begin{description}
\htmlDD The resource broker to access the cms.

\end{description}

o {\bf m\_launcherManager} 

\begin{PRE}
 private CmsLauncherManager m\_launcherManager
\end{PRE}

\begin{description}
\htmlDD The launcher manager used with this object, Is needed to clear the
template caches.

\end{description}

o {\bf m\_linkSubstitution} 

\begin{PRE}
 private LinkSubstitution m\_linkSubstitution
\end{PRE}

\begin{description}
\htmlDD The class for processing links.

\end{description}

o {\bf m\_mode} 

\begin{PRE}
 private int m\_mode
\end{PRE}

\begin{description}
\htmlDD the modus the cmsObject runs in (used i.e. for static export)

\end{description}

\subsection*{  Constructors }

o {\bf CmsObject} 

\begin{PRE}
 public CmsObject()
\end{PRE}

\begin{description}
\htmlDD The default constructor. 

\end{description}

\subsection*{  Methods }

o {\bf acceptTask} 

\begin{PRE}
 public void acceptTask(int taskId) throws CmsException
\end{PRE}

\begin{description}
\htmlDD Accept a task from the Cms. 

\begin{description}
\item {\bf Parameters:}  

taskid - the id of the task to accept.  
\item {\bf Throws:} CmsException  

if operation was not successful.  
\end{description}

\end{description}

o {\bf accessCreate} 

\begin{PRE}
 public boolean accessCreate(String resource) throws CmsException
\end{PRE}

\begin{description}
\htmlDD Checks, if the user may create this resource. 

\begin{description}
\item {\bf Parameters:}  

resource - the resource to check.  
\item {\bf Returns:}  

{\tt true} if the user has the appropriate rigths to create the resource; {\tt
false} otherwise  
\item {\bf Throws:} CmsException  

if operation was not successful.  
\end{description}

\end{description}

o {\bf accessLock} 

\begin{PRE}
 public boolean accessLock(String resource) throws CmsException
\end{PRE}

\begin{description}
\htmlDD Checks, if the user may lock this resource. 

\begin{description}
\item {\bf Parameters:}  

resource - the resource to check.  
\item {\bf Returns:}  

{\tt true} if the user has the appropriate rights to lock this resource; {\tt
false} otherwise  
\item {\bf Throws:} CmsException  

if operation was not successful.  
\end{description}

\end{description}

o {\bf accessProject} 

\begin{PRE}
 public boolean accessProject(int projectId) throws CmsException
\end{PRE}

\begin{description}
\htmlDD Checks if the user can access the project. 

\begin{description}
\item {\bf Parameters:}  

projectId - the id of the project.  
\item {\bf Returns:}  

{\tt true}, if the user may access this project; {\tt false} otherwise  
\item {\bf Throws:} CmsException  

if operation was not successful.  
\end{description}

\end{description}

o {\bf accessRead} 

\begin{PRE}
 public boolean accessRead(String resource) throws CmsException
\end{PRE}

\begin{description}
\htmlDD Checks, if the user may read this resource. 

\begin{description}
\item {\bf Parameters:}  

resource - The resource to check.  
\item {\bf Returns:}  

{\tt true}, if the user has the appropriate rigths to read the resource; {\tt
false} otherwise.  
\item {\bf Throws:} CmsException  

if operation was not successful.  
\end{description}

\end{description}

o {\bf accessWrite} 

\begin{PRE}
 public boolean accessWrite(String resource) throws CmsException
\end{PRE}

\begin{description}
\htmlDD Checks, if the user may write this resource. 

\begin{description}
\item {\bf Parameters:}  

resource - the resource to check.  
\item {\bf Returns:}  

{\tt true}, if the user has the appropriate rigths to write the resource; {\tt
false} otherwise.  
\item {\bf Throws:} CmsException  

if operation was not successful.  
\end{description}

\end{description}

o {\bf addFileExtension} 

\begin{PRE}
 public void addFileExtension(String extension,
                              String resTypeName) throws CmsException
\end{PRE}

\begin{description}
\htmlDD Adds a file extension to the list of known file extensions. 

{\bf Security:} Only members of the group administrators are allowed to add a
file extension. 

\begin{description}
\item {\bf Parameters:}  

extension - a file extension like ``html'',``txt'' etc.  

resTypeName - name of the resource type associated with the extension.  
\item {\bf Throws:} CmsException  

if operation was not successful.  
\end{description}

\end{description}

o {\bf addGroup} 

\begin{PRE}
 public CmsGroup addGroup(String name,
                          String description,
                          int flags,
                          String parent) throws CmsException
\end{PRE}

\begin{description}
\htmlDD Adds a new group to the Cms. 

{\bf Security:} Only members of the group administrators are allowed to add a
new group. 

\begin{description}
\item {\bf Parameters:}  

name - the name of the new group.  

description - the description of the new group.  
\item {\bf Returns:}  

a {\tt CmsGroup} object representing the newly created group.  
\item {\bf Throws:} CmsException  

if operation was not successful.  
\end{description}

\end{description}

o {\bf addUser} 

\begin{PRE}
 public CmsUser addUser(String name,
                        String password,
                        String group,
                        String description,
                        Hashtable additionalInfos,
                        int flags) throws CmsException
\end{PRE}

\begin{description}
\htmlDD Adds a user to the Cms. 

{\bf Security:} Only members of the group administrators are allowed to add a
user. 

\begin{description}
\item {\bf Parameters:}  

name - the new name for the user.  

password - the new password for the user.  

group - the default groupname for the user.  

description - the description for the user.  

additionalInfos - a Hashtable with additional infos for the user. These Infos
may be stored into the Usertables (depending on the implementation).  

flags - the flags for a user (e.g. C\_FLAG\_ENABLED).  
\item {\bf Returns:}  

a {\tt CmsUser} object representing the added user.  
\item {\bf Throws:} CmsException  

if operation was not successful.  
\end{description}

\end{description}

o {\bf addImportUser} 

\begin{PRE}
 public CmsUser addImportUser(String name,
                              String password,
                              String recoveryPassword,
                              String description,
                              String firstname,
                              String lastname,
                              String email,
                              int flags,
                              Hashtable additionalInfos,
                              String defaultGroup,
                              String address,
                              String section,
                              int type) throws CmsException
\end{PRE}

\begin{description}
\htmlDD Adds a user to the Cms by import. 

{\bf Security:} Only members of the group administrators are allowed to add a
user. 

\begin{description}
\item {\bf Parameters:}  

name - the new name for the user.  

password - the new password for the user.  

recoveryPassword - the new password for the user.  

description - the description for the user.  

firstname - the firstname of the user.  

lastname - the lastname of the user.  

email - the email of the user.  

flags - the flags for a user (e.g. C\_FLAG\_ENABLED).  

additionalInfos - a Hashtable with additional infos for the user. These Infos
may be stored into the Usertables (depending on the implementation).  

defaultGroup - the default groupname for the user.  

address - the address of the user.  

section - the section of the user.  

type - the type of the user.  
\item {\bf Returns:}  

a {\tt CmsUser} object representing the added user.  
\item {\bf Throws:} CmsException  

if operation was not successful.  
\end{description}

\end{description}

o {\bf addUserToGroup} 

\begin{PRE}
 public void addUserToGroup(String username,
                            String groupname) throws CmsException
\end{PRE}

\begin{description}
\htmlDD Adds a user to a group. 

{\bf Security:} Only members of the group administrators are allowed to add a
user to a group. 

\begin{description}
\item {\bf Parameters:}  

username - the name of the user that is to be added to the group.  

groupname - the name of the group.  
\item {\bf Throws:} CmsException  

if operation was not successful.  
\end{description}

\end{description}

o {\bf addWebUser} 

\begin{PRE}
 public CmsUser addWebUser(String name,
                           String password,
                           String group,
                           String description,
                           Hashtable additionalInfos,
                           int flags) throws CmsException
\end{PRE}

\begin{description}
\htmlDD Adds a web user to the Cms. \htmlBR
A web user has no access to the workplace but is able to access personalized
functions controlled by the OpenCms. 

\begin{description}
\item {\bf Parameters:}  

name - the new name for the user.  

password - the new password for the user.  

group - the default groupname for the user.  

description - the description for the user.  

additionalInfos - a Hashtable with additional infos for the user. These Infos
may be stored into the Usertables (depending on the implementation).  

flags - the flags for a user (e.g. C\_FLAG\_ENABLED)  
\item {\bf Returns:}  

a {\tt CmsUser} object representing the newly created user.  
\item {\bf Throws:} CmsException  

if operation was not successful.  
\end{description}

\end{description}

o {\bf addWebUser} 

\begin{PRE}
 public CmsUser addWebUser(String name,
                           String password,
                           String group,
                           String additionalGroup,
                           String description,
                           Hashtable additionalInfos,
                           int flags) throws CmsException
\end{PRE}

\begin{description}
\htmlDD Adds a web user to the Cms. \htmlBR
A web user has no access to the workplace but is able to access personalized
functions controlled by the OpenCms. 

\begin{description}
\item {\bf Parameters:}  

name - the new name for the user.  

password - the new password for the user.  

group - the default groupname for the user.  

additionalGroup - An additional group for the user.  

description - the description for the user.  

additionalInfos - a Hashtable with additional infos for the user. These Infos
may be stored into the Usertables (depending on the implementation).  

flags - the flags for a user (e.g. C\_FLAG\_ENABLED)  
\item {\bf Returns:}  

a {\tt CmsUser} object representing the newly created user.  
\item {\bf Throws:} CmsException  

if operation was not successful.  
\end{description}

\end{description}

o {\bf anonymousUser} 

\begin{PRE}
 public CmsUser anonymousUser() throws CmsException
\end{PRE}

\begin{description}
\htmlDD Returns the anonymous user object. 

\begin{description}
\item {\bf Returns:}  

a {\tt CmsUser} object representing the anonymous user.  
\item {\bf Throws:} CmsException  

if operation was not successful.  
\end{description}

\end{description}

o {\bf chgrp} 

\begin{PRE}
 public void chgrp(String filename,
                   String newGroup) throws CmsException
\end{PRE}

\begin{description}
\htmlDD Changes the group of a resource. \htmlBR
Only the group of a resource in an offline project can be changed. The state
of the resource is set to CHANGED (1). If the content of this resource is not
existing in the offline project already, it is read from the online project
and written into the offline project. 

{\bf Security:} Access is granted, if: 

\begin{itemize}
\item the user has access to the project 
\item the user is owner of the resource or is admin 
\item the resource is locked by the callingUser 
\end{itemize}

\begin{description}
\item {\bf Parameters:}  

filename - the complete path to the resource.  

newGroup - the name of the new group for this resource.  
\item {\bf Throws:} CmsException  

if operation was not successful.  
\end{description}

\end{description}

o {\bf chgrp} 

\begin{PRE}
 public void chgrp(String filename,
                   String newGroup,
                   boolean chRekursive) throws CmsException
\end{PRE}

\begin{description}
\htmlDD Changes the group of a resource. \htmlBR
Only the group of a resource in an offline project can be changed. The state
of the resource is set to CHANGED (1). If the content of this resource is not
existing in the offline project already, it is read from the online project
and written into the offline project. 

{\bf Security:} Access is granted, if: 

\begin{itemize}
\item the user has access to the project 
\item the user is owner of the resource or is admin 
\item the resource is locked by the callingUser 
\end{itemize}

\begin{description}
\item {\bf Parameters:}  

filename - the complete path to the resource.  

newGroup - the name of the new group for this resource.  

chRekursive - shows if the subResources (of a folder) should be changed too.  
\item {\bf Throws:} CmsException  

if operation was not successful.  
\end{description}

\end{description}

o {\bf doChgrp} 

\begin{PRE}
 protected void doChgrp(String filename,
                        String newGroup) throws CmsException
\end{PRE}

\begin{description}
\htmlDD Changes the group of a resource. \htmlBR
Only the group of a resource in an offline project can be changed. The state
of the resource is set to CHANGED (1). If the content of this resource is not
existing in the offline project already, it is read from the online project
and written into the offline project. 

{\bf Security:} Access is granted, if: 

\begin{itemize}
\item the user has access to the project 
\item the user is owner of the resource or is admin 
\item the resource is locked by the callingUser 
\end{itemize}

\begin{description}
\item {\bf Parameters:}  

filename - the complete path to the resource.  

newGroup - the name of the new group for this resource.  
\item {\bf Throws:} CmsException  

if operation was not successful.  
\end{description}

\end{description}

o {\bf chmod} 

\begin{PRE}
 public void chmod(String filename,
                   int flags) throws CmsException
\end{PRE}

\begin{description}
\htmlDD Changes the flags of a resource. \htmlBR
Only the flags of a resource in an offline project can be changed. The state
of the resource is set to CHANGED (1). If the content of this resource is not
existing in the offline project already, it is read from the online project
and written into the offline project. The user may change the flags, if he is
admin of the resource. 

{\bf Security:} Access is granted, if: 

\begin{itemize}
\item the user has access to the project 
\item the user can write the resource 
\item the resource is locked by the callingUser 
\end{itemize}

\begin{description}
\item {\bf Parameters:}  

filename - the complete path to the resource.  

flags - the new flags for the resource.  
\item {\bf Throws:} CmsException  

if operation was not successful. for this resource.  
\end{description}

\end{description}

o {\bf chmod} 

\begin{PRE}
 public void chmod(String filename,
                   int flags,
                   boolean chRekursive) throws CmsException
\end{PRE}

\begin{description}
\htmlDD Changes the flags of a resource. \htmlBR
Only the flags of a resource in an offline project can be changed. The state
of the resource is set to CHANGED (1). If the content of this resource is not
existing in the offline project already, it is read from the online project
and written into the offline project. The user may change the flags, if he is
admin of the resource. 

{\bf Security:} Access is granted, if: 

\begin{itemize}
\item the user has access to the project 
\item the user can write the resource 
\item the resource is locked by the callingUser 
\end{itemize}

\begin{description}
\item {\bf Parameters:}  

filename - the complete path to the resource.  

flags - the new flags for the resource.  

chRekursive - shows if the subResources (of a folder) should be changed too.  
\item {\bf Throws:} CmsException  

if operation was not successful. for this resource.  
\end{description}

\end{description}

o {\bf doChmod} 

\begin{PRE}
 protected void doChmod(String filename,
                        int flags) throws CmsException
\end{PRE}

\begin{description}
\htmlDD Changes the flags of a resource. \htmlBR
Only the flags of a resource in an offline project can be changed. The state
of the resource is set to CHANGED (1). If the content of this resource is not
existing in the offline project already, it is read from the online project
and written into the offline project. The user may change the flags, if he is
admin of the resource. 

{\bf Security:} Access is granted, if: 

\begin{itemize}
\item the user has access to the project 
\item the user can write the resource 
\item the resource is locked by the callingUser 
\end{itemize}

\begin{description}
\item {\bf Parameters:}  

filename - the complete path to the resource.  

flags - the new flags for the resource.  
\item {\bf Throws:} CmsException  

if operation was not successful. for this resource.  
\end{description}

\end{description}

o {\bf chown} 

\begin{PRE}
 public void chown(String filename,
                   String newOwner) throws CmsException
\end{PRE}

\begin{description}
\htmlDD Changes the owner of a resource. \htmlBR
Only the owner of a resource in an offline project can be changed. The state
of the resource is set to CHANGED (1). If the content of this resource is not
existing in the offline project already, it is read from the online project
and written into the offline project. The user may change this, if he is admin
of the resource. 

{\bf Security:} Access is cranted, if: 

\begin{itemize}
\item the user has access to the project 
\item the user is owner of the resource or the user is admin 
\item the resource is locked by the callingUser 
\end{itemize}

\begin{description}
\item {\bf Parameters:}  

filename - the complete path to the resource.  

newOwner - the name of the new owner for this resource.  
\item {\bf Throws:} CmsException  

if operation was not successful.  
\end{description}

\end{description}

o {\bf chown} 

\begin{PRE}
 public void chown(String filename,
                   String newOwner,
                   boolean chRekursive) throws CmsException
\end{PRE}

\begin{description}
\htmlDD Changes the owner of a resource. \htmlBR
Only the owner of a resource in an offline project can be changed. The state
of the resource is set to CHANGED (1). If the content of this resource is not
existing in the offline project already, it is read from the online project
and written into the offline project. The user may change this, if he is admin
of the resource. 

{\bf Security:} Access is cranted, if: 

\begin{itemize}
\item the user has access to the project 
\item the user is owner of the resource or the user is admin 
\item the resource is locked by the callingUser 
\end{itemize}

\begin{description}
\item {\bf Parameters:}  

filename - the complete path to the resource.  

newOwner - the name of the new owner for this resource.  

chRekursive - shows if the subResources (of a folder) should be changed too.  
\item {\bf Throws:} CmsException  

if operation was not successful.  
\end{description}

\end{description}

o {\bf doChown} 

\begin{PRE}
 protected void doChown(String filename,
                        String newOwner) throws CmsException
\end{PRE}

\begin{description}
\htmlDD Changes the owner of a resource. \htmlBR
Only the owner of a resource in an offline project can be changed. The state
of the resource is set to CHANGED (1). If the content of this resource is not
existing in the offline project already, it is read from the online project
and written into the offline project. The user may change this, if he is admin
of the resource. 

{\bf Security:} Access is cranted, if: 

\begin{itemize}
\item the user has access to the project 
\item the user is owner of the resource or the user is admin 
\item the resource is locked by the callingUser 
\end{itemize}

\begin{description}
\item {\bf Parameters:}  

filename - the complete path to the resource.  

newOwner - the name of the new owner for this resource.  
\item {\bf Throws:} CmsException  

if operation was not successful.  
\end{description}

\end{description}

o {\bf chtype} 

\begin{PRE}
 public void chtype(String filename,
                    String newType) throws CmsException
\end{PRE}

\begin{description}
\htmlDD Changes the resourcetype of a resource. \htmlBR
Only the resourcetype of a resource in an offline project can be changed. The
state of the resource is set to CHANGED (1). If the content of this resource
is not exisiting in the offline project already, it is read from the online
project and written into the offline project. The user may change this, if he
is admin of the resource. 

{\bf Security:} Access is granted, if: 

\begin{itemize}
\item the user has access to the project 
\item the user is owner of the resource or is admin 
\item the resource is locked by the callingUser 
\end{itemize}

\begin{description}
\item {\bf Parameters:}  

filename - the complete path to the resource.  

newType - the name of the new resourcetype for this resource.  
\item {\bf Throws:} CmsException  

if operation was not successful.  
\end{description}

\end{description}

o {\bf doChtype} 

\begin{PRE}
 protected void doChtype(String filename,
                         String newType) throws CmsException
\end{PRE}

\begin{description}
\htmlDD Changes the resourcetype of a resource. \htmlBR
Only the resourcetype of a resource in an offline project can be changed. The
state of the resource is set to CHANGED (1). If the content of this resource
is not exisiting in the offline project already, it is read from the online
project and written into the offline project. The user may change this, if he
is admin of the resource. 

{\bf Security:} Access is granted, if: 

\begin{itemize}
\item the user has access to the project 
\item the user is owner of the resource or is admin 
\item the resource is locked by the callingUser 
\end{itemize}

\begin{description}
\item {\bf Parameters:}  

filename - the complete path to the resource.  

newType - the name of the new resourcetype for this resource.  
\item {\bf Throws:} CmsException  

if operation was not successful.  
\end{description}

\end{description}

o {\bf clearcache} 

\begin{PRE}
 public void clearcache()
\end{PRE}

\begin{description}
\htmlDD Clears all internal DB-Caches. 

\end{description}

o {\bf copyResource} 

\begin{PRE}
 public void copyResource(String source,
                          String destination) throws CmsException
\end{PRE}

\begin{description}
\htmlDD Copies a file. 

\begin{description}
\item {\bf Parameters:}  

source - the complete path of the sourcefile.  

destination - the complete path of the destinationfolder.  
\item {\bf Throws:} CmsException  

if the file couldn't be copied, or the user has not the appropriate rights to
copy the file.  
\end{description}

\end{description}

o {\bf copyResource} 

\begin{PRE}
 public void copyResource(String source,
                          String destination,
                          boolean keepFlags) throws CmsException
\end{PRE}

\begin{description}
\htmlDD Copies a file. 

\begin{description}
\item {\bf Parameters:}  

source - the complete path of the sourcefile.  

destination - the complete path of the destinationfolder.  

keepFlags - {\tt true} if the copy should keep the source file's flags,  {\tt
false} if the copy should get the user's default flags.  
\item {\bf Throws:} CmsException  

if the file couldn't be copied, or the user has not the appropriate rights to
copy the file.  
\end{description}

\end{description}

o {\bf doCopyFile} 

\begin{PRE}
 protected void doCopyFile(String source,
                           String destination) throws CmsException
\end{PRE}

\begin{description}
\htmlDD Copies a file. 

\begin{description}
\item {\bf Parameters:}  

source - the complete path of the sourcefile.  

destination - the complete path of the destinationfolder.  
\item {\bf Throws:} CmsException  

if the file couldn't be copied, or the user has not the appropriate rights to
copy the file.  
\end{description}

\end{description}

o {\bf doCopyFolder} 

\begin{PRE}
 protected void doCopyFolder(String source,
                             String destination) throws CmsException
\end{PRE}

\begin{description}
\htmlDD Copies a folder. 

\begin{description}
\item {\bf Parameters:}  

source - the complete path of the sourcefolder.  

destination - the complete path of the destinationfolder.  
\item {\bf Throws:} CmsException  

if the folder couldn't be copied, or if the user has not the appropriate
rights to copy the folder.  
\end{description}

\end{description}

o {\bf copyFile} 

\begin{PRE}
 public void copyFile(String source,
                      String destination) throws CmsException
\end{PRE}

\begin{description}
\htmlDD {\bf Note: copyFile() is deprecated.} {\it Use copyResource instead.} 

Copies a file. 

\begin{description}
\item {\bf Parameters:}  

source - the complete path of the sourcefile.  

destination - the complete path of the destinationfolder.  
\item {\bf Throws:} CmsException  

if the file couldn't be copied, or the user has not the appropriate rights to
copy the file.  
\end{description}

\end{description}

o {\bf copyFolder} 

\begin{PRE}
 public void copyFolder(String source,
                        String destination) throws CmsException
\end{PRE}

\begin{description}
\htmlDD {\bf Note: copyFolder() is deprecated.} {\it Use copyResource
instead.} 

Copies a folder. 

\begin{description}
\item {\bf Parameters:}  

source - the complete path of the sourcefolder.  

destination - the complete path of the destinationfolder.  
\item {\bf Throws:} CmsException  

if the folder couldn't be copied, or if the user has not the appropriate
rights to copy the folder.  
\end{description}

\end{description}

o {\bf copyResourceToProject} 

\begin{PRE}
 public void copyResourceToProject(String resource) throws CmsException
\end{PRE}

\begin{description}
\htmlDD Copies a resource from the online project to a new, specified project.
\htmlBR
Copying a resource will copy the file header or folder into the specified
offline project and set its state to UNCHANGED. 

\begin{description}
\item {\bf Parameters:}  

resource - the name of the resource.  
\item {\bf Throws:} CmsException  

if operation was not successful.  
\end{description}

\end{description}

o {\bf doCopyResourceToProject} 

\begin{PRE}
 protected void doCopyResourceToProject(String resource) throws CmsException
\end{PRE}

\begin{description}
\htmlDD Copies a resource from the online project to a new, specified project.
\htmlBR
Copying a resource will copy the file header or folder into the specified
offline project and set its state to UNCHANGED. 

\begin{description}
\item {\bf Parameters:}  

resource - the name of the resource.  
\item {\bf Throws:} CmsException  

if operation was not successful.  
\end{description}

\end{description}

o {\bf copyright} 

\begin{PRE}
 public String[] copyright()
\end{PRE}

\begin{description}
\htmlDD Returns the copyright information for this OpenCms. 

\begin{description}
\item {\bf Returns:}  

copyright a String arry containing copyright information.  
\end{description}

\end{description}

o {\bf countLockedResources} 

\begin{PRE}
 public int countLockedResources(int id) throws CmsException
\end{PRE}

\begin{description}
\htmlDD Counts the locked resources in a project. 

\begin{description}
\item {\bf Parameters:}  

id - the id of the project  
\item {\bf Returns:}  

the number of locked resources in this project.  
\item {\bf Throws:} CmsException  

if operation was not successful.  
\end{description}

\end{description}

o {\bf createFile} 

\begin{PRE}
 public CmsFile createFile(String folder,
                           String filename,
                           byte contents[],
                           String type) throws CmsException
\end{PRE}

\begin{description}
\htmlDD {\bf Note: createFile() is deprecated.} {\it Use createResource
instead.} 

Creates a new file with the given content and resourcetype.\htmlBR

\begin{description}
\item {\bf Parameters:}  

folder - the complete path to the folder in which the file will be created.  

filename - the name of the new file.  

contents - the contents of the new file.  

type - the resourcetype of the new file.  
\item {\bf Returns:}  

file a {\tt CmsFile} object representing the newly created file.  
\item {\bf Throws:} if  

the resourcetype is set to folder. The CmsException is also thrown, if the
filename is not valid or if the user has not the appropriate rights to create
a new file.  
\end{description}

\end{description}

o {\bf createFile} 

\begin{PRE}
 public CmsFile createFile(String folder,
                           String filename,
                           byte contents[],
                           String type,
                           Hashtable properties) throws CmsException
\end{PRE}

\begin{description}
\htmlDD {\bf Note: createFile() is deprecated.} {\it Use createResource
instead.} 

Creates a new file with the given content and resourcetype. 

\begin{description}
\item {\bf Parameters:}  

folder - the complete path to the folder in which the file will be created.  

filename - the name of the new file.  

contents - the contents of the new file.  

type - the resourcetype of the new file.  

properties - A Hashtable of properties, that should be set for this file. The
keys for this Hashtable are the names for properties, the values are the
values for the properties.  
\item {\bf Returns:}  

file a {\tt CmsFile} object representing the newly created file.  
\item {\bf Throws:} CmsException  

or if the resourcetype is set to folder. The CmsException is also thrown, if
the filename is not valid or if the user has not the appropriate rights to
create a new file.  
\end{description}

\end{description}

o {\bf createFolder} 

\begin{PRE}
 public CmsFolder createFolder(String folder,
                               String newFolderName) throws CmsException
\end{PRE}

\begin{description}
\htmlDD {\bf Note: createFolder() is deprecated.} {\it Use createResource
instead.} 

Creates a new folder. 

\begin{description}
\item {\bf Parameters:}  

folder - the complete path to the folder in which the new folder will be
created.  

newFolderName - the name of the new folder.  
\item {\bf Returns:}  

folder a {\tt CmsFolder} object representing the newly created folder.  
\item {\bf Throws:} CmsException  

if the foldername is not valid, or if the user has not the appropriate rights
to create a new folder.  
\end{description}

\end{description}

o {\bf createChannel} 

\begin{PRE}
 public CmsFolder createChannel(String parentChannel,
                                String newChannelName) throws CmsException
\end{PRE}

\begin{description}
\htmlDD Creates a new channel. 

\begin{description}
\item {\bf Parameters:}  

parentChannel - the complete path to the channel in which the new channel will
be created.  

newChannelName - the name of the new channel.  
\item {\bf Returns:}  

folder a {\tt CmsFolder} object representing the newly created channel.  
\item {\bf Throws:} CmsException  

if the channelname is not valid, or if the user has not the appropriate rights
to create a new channel.  
\end{description}

\end{description}

o {\bf createFolder} 

\begin{PRE}
 public CmsFolder createFolder(String folder,
                               String newFolderName,
                               Hashtable properties) throws CmsException
\end{PRE}

\begin{description}
\htmlDD {\bf Note: createFolder() is deprecated.} {\it Use createResource
instead.} 

Creates a new folder. 

\begin{description}
\item {\bf Parameters:}  

folder - the complete path to the folder in which the new folder will be
created.  

newFolderName - the name of the new folder.  

properties - A Hashtable of properties, that should be set for this folder.
The keys for this Hashtable are the names for property-definitions, the values
are the values for the properties.  
\item {\bf Returns:}  

a {\tt CmsFolder} object representing the newly created folder.  
\item {\bf Throws:} CmsException  

if the foldername is not valid, or if the user has not the appropriate rights
to create a new folder.  
\end{description}

\end{description}

o {\bf createResource} 

\begin{PRE}
 public CmsResource createResource(String folder,
                                   String name,
                                   String type) throws CmsException
\end{PRE}

o {\bf createResource} 

\begin{PRE}
 public CmsResource createResource(String folder,
                                   String name,
                                   String type,
                                   Hashtable properties) throws CmsException
\end{PRE}

o {\bf createResource} 

\begin{PRE}
 public CmsResource createResource(String folder,
                                   String name,
                                   String type,
                                   Hashtable properties,
                                   byte contents[]) throws CmsException
\end{PRE}

o {\bf doCreateFile} 

\begin{PRE}
 protected CmsFile doCreateFile(String folder,
                                String filename,
                                byte contents[],
                                String type) throws CmsException
\end{PRE}

\begin{description}
\htmlDD Creates a new file with the given content and resourcetype.\htmlBR

\begin{description}
\item {\bf Parameters:}  

folder - the complete path to the folder in which the file will be created.  

filename - the name of the new file.  

contents - the contents of the new file.  

type - the resourcetype of the new file.  
\item {\bf Returns:}  

file a {\tt CmsFile} object representing the newly created file.  
\item {\bf Throws:} CmsException  

if the resourcetype is set to folder. The CmsException is also thrown, if the
filename is not valid or if the user has not the appropriate rights to create
a new file.  
\end{description}

\end{description}

o {\bf doCreateFile} 

\begin{PRE}
 protected CmsFile doCreateFile(String folder,
                                String filename,
                                byte contents[],
                                String type,
                                Hashtable properties) throws CmsException
\end{PRE}

\begin{description}
\htmlDD Creates a new file with the given content and resourcetype. 

\begin{description}
\item {\bf Parameters:}  

folder - the complete path to the folder in which the file will be created.  

filename - the name of the new file.  

contents - the contents of the new file.  

type - the resourcetype of the new file.  

properties - A Hashtable of properties, that should be set for this file. The
keys for this Hashtable are the names for properties, the values are the
values for the properties.  
\item {\bf Returns:}  

file a {\tt CmsFile} object representing the newly created file.  
\item {\bf Throws:} CmsException  

if the wrong properties are given, or if the resourcetype is set to folder.
The CmsException is also thrown, if the filename is not valid or if the user
has not the appropriate rights to create a new file.  
\end{description}

\end{description}

o {\bf doCreateFolder} 

\begin{PRE}
 protected CmsFolder doCreateFolder(String folder,
                                    String newFolderName) throws CmsException
\end{PRE}

\begin{description}
\htmlDD Creates a new folder. 

\begin{description}
\item {\bf Parameters:}  

folder - the complete path to the folder in which the new folder will be
created.  

newFolderName - the name of the new folder.  
\item {\bf Returns:}  

folder a {\tt CmsFolder} object representing the newly created folder.  
\item {\bf Throws:} CmsException  

if the foldername is not valid, or if the user has not the appropriate rights
to create a new folder.  
\end{description}

\end{description}

o {\bf doCreateFolder} 

\begin{PRE}
 protected CmsFolder doCreateFolder(String folder,
                                    String newFolderName,
                                    Hashtable properties) throws CmsException
\end{PRE}

\begin{description}
\htmlDD Creates a new folder. 

\begin{description}
\item {\bf Parameters:}  

folder - the complete path to the folder in which the new folder will be
created.  

newFolderName - the name of the new folder.  

properties - A Hashtable of properties, that should be set for this folder.
The keys for this Hashtable are the names for property-definitions, the values
are the values for the properties.  
\item {\bf Returns:}  

a {\tt CmsFolder} object representing the newly created folder.  
\item {\bf Throws:} CmsException  

if the foldername is not valid, or if the user has not the appropriate rights
to create a new folder.  
\end{description}

\end{description}

o {\bf createProject} 

\begin{PRE}
 public CmsTask createProject(String projectname,
                              int projectType,
                              String roleName,
                              long timeout,
                              int priority) throws CmsException
\end{PRE}

\begin{description}
\htmlDD Creates a new project for task handling. 

\begin{description}
\item {\bf Parameters:}  

projectname - the name of the project  

projectType - the type of the Project  

role - a Usergroup for the project  

timeout - the time when the Project must finished  

priority - a Priority for the Project  
\item {\bf Returns:}  

a {\tt CmsTask} object representing the newly created task.  
\item {\bf Throws:} CmsException  

if operation was not successful.  
\end{description}

\end{description}

o {\bf createProject} 

\begin{PRE}
 public CmsProject createProject(String name,
                                 String description,
                                 String groupname,
                                 String managergroupname) throws CmsException
\end{PRE}

\begin{description}
\htmlDD Creates a new project. 

\begin{description}
\item {\bf Parameters:}  

name - the name of the project to read.  

description - the description for the new project.  

groupname - the name of the group to be set.  

managergroupname - the name of the managergroup to be set.  
\item {\bf Throws:} CmsException  

if operation was not successful.  
\end{description}

\end{description}

o {\bf createProject} 

\begin{PRE}
 public CmsProject createProject(String name,
                                 String description,
                                 String groupname,
                                 String managergroupname,
                                 int projecttype) throws CmsException
\end{PRE}

\begin{description}
\htmlDD Creates a new project. 

\begin{description}
\item {\bf Parameters:}  

name - the name of the project to read.  

description - the description for the new project.  

groupname - the name of the group to be set.  

managergroupname - the name of the managergroup to be set.  

projecttype - the type of the project (normal or temporary)  
\item {\bf Throws:} CmsException  

if operation was not successful.  
\end{description}

\end{description}

o {\bf createTempfileProject} 

\begin{PRE}
 public CmsProject createTempfileProject() throws CmsException
\end{PRE}

\begin{description}
\htmlDD Creates a new project for the temporary files. 

\begin{description}
\item {\bf Throws:} CmsException  

if operation was not successful.  
\end{description}

\end{description}

o {\bf createPropertydefinition} 

\begin{PRE}
 public CmsPropertydefinition createPropertydefinition(String name,
                                                       String resourcetype,
                                                       int type) throws CmsException
\end{PRE}

\begin{description}
\htmlDD {\bf Note: createPropertydefinition() is deprecated.} {\it Use
createPropertydefinition without type of propertydefinition instead.} 

Creates the property-definition for a resource type. 

\begin{description}
\item {\bf Parameters:}  

name - the name of the property-definition to overwrite.  

resourcetype - the name of the resource-type for the property-definition.  

type - the type of the property-definition (normal{\htmlBar}optional)  
\item {\bf Throws:} CmsException  

if operation was not successful.  
\end{description}

\end{description}

o {\bf createPropertydefinition} 

\begin{PRE}
 public CmsPropertydefinition createPropertydefinition(String name,
                                                       String resourcetype) throws CmsException
\end{PRE}

\begin{description}
\htmlDD Creates the property-definition for a resource type. 

\begin{description}
\item {\bf Parameters:}  

name - the name of the property-definition to overwrite.  

resourcetype - the name of the resource-type for the property-definition.  
\item {\bf Throws:} CmsException  

if operation was not successful.  
\end{description}

\end{description}

o {\bf createTask} 

\begin{PRE}
 public CmsTask createTask(int projectid,
                           String agentName,
                           String roleName,
                           String taskname,
                           String taskcomment,
                           int tasktype,
                           long timeout,
                           int priority) throws CmsException
\end{PRE}

\begin{description}
\htmlDD Creates a new task. 

{\bf Security:} All users can create a new task. 

\begin{description}
\item {\bf Parameters:}  

projectid - the Id of the current project task of the user.  

agentname - the User who will edit the task.  

rolename - a Usergroup for the task.  

taskname - a Name of the task.  

tasktype - the type of the task.  

taskcomment - a description of the task.  

timeout - the time when the task must finished.  

priority - the Id for the priority of the task.  
\item {\bf Returns:}  

a {\tt CmsTask} object representing the newly created task.  
\item {\bf Throws:} CmsException  

Throws CmsException if something goes wrong.  
\end{description}

\end{description}

o {\bf createTask} 

\begin{PRE}
 public CmsTask createTask(String agentName,
                           String roleName,
                           String taskname,
                           String taskcomment,
                           long timeout,
                           int priority) throws CmsException
\end{PRE}

\begin{description}
\htmlDD Creates a new task. 

{\bf Security:} All users can create a new task. 

\begin{description}
\item {\bf Parameters:}  

agent - the User who will edit the task.  

role - a Usergroup for the task.  

taskname - the name of the task.  

taskcomment - a description of the task.  

timeout - the time when the task must finished.  

priority - the Id for the priority of the task.  
\item {\bf Returns:}  

a {\tt CmsTask} object representing the newly created task.  
\item {\bf Throws:} CmsException  

if operation was not successful.  
\end{description}

\end{description}

o {\bf deleteAllProperties} 

\begin{PRE}
 public void deleteAllProperties(String resourcename) throws CmsException
\end{PRE}

\begin{description}
\htmlDD Deletes all properties for a file or folder. 

\begin{description}
\item {\bf Parameters:}  

resourcename - the name of the resource for which all properties should be
deleted.  
\item {\bf Throws:} CmsException  

if operation was not successful.  
\end{description}

\end{description}

o {\bf deleteFile} 

\begin{PRE}
 public void deleteFile(String filename) throws CmsException
\end{PRE}

\begin{description}
\htmlDD {\bf Note: deleteFile() is deprecated.} {\it Use deleteResource
instead.} 

Deletes a file. 

\begin{description}
\item {\bf Parameters:}  

filename - the complete path of the file.  
\item {\bf Throws:} CmsException  

if the file couldn't be deleted, or if the user has not the appropriate rights
to delete the file.  
\end{description}

\end{description}

o {\bf deleteFolder} 

\begin{PRE}
 public void deleteFolder(String foldername) throws CmsException
\end{PRE}

\begin{description}
\htmlDD {\bf Note: deleteFolder() is deprecated.} {\it Use deleteResource
instead.} 

Deletes a folder. \htmlBR
This is a very complex operation, because all sub-resources may be deleted
too. 

\begin{description}
\item {\bf Parameters:}  

foldername - the complete path of the folder.  
\item {\bf Throws:} CmsException  

if the folder couldn't be deleted, or if the user has not the rights to delete
this folder.  
\end{description}

\end{description}

o {\bf deleteEmptyFolder} 

\begin{PRE}
 public void deleteEmptyFolder(String foldername) throws CmsException
\end{PRE}

\begin{description}
\htmlDD Deletes a folder. \htmlBR
This is a very complex operation, because all sub-resources may be deleted
too. 

\begin{description}
\item {\bf Parameters:}  

foldername - the complete path of the folder.  
\item {\bf Throws:} CmsException  

if the folder couldn't be deleted, or if the user has not the rights to delete
this folder.  
\end{description}

\end{description}

o {\bf deleteResource} 

\begin{PRE}
 public void deleteResource(String filename) throws CmsException
\end{PRE}

\begin{description}
\htmlDD Deletes a resource. 

\begin{description}
\item {\bf Parameters:}  

filename - the complete path of the file.  
\item {\bf Throws:} CmsException  

if the file couldn't be deleted, or if the user has not the appropriate rights
to delete the file.  
\end{description}

\end{description}

o {\bf doDeleteFile} 

\begin{PRE}
 protected void doDeleteFile(String filename) throws CmsException
\end{PRE}

\begin{description}
\htmlDD Deletes a file. 

\begin{description}
\item {\bf Parameters:}  

filename - the complete path of the file.  
\item {\bf Throws:} CmsException  

if the file couldn't be deleted, or if the user has not the appropriate rights
to delete the file.  
\end{description}

\end{description}

o {\bf doDeleteFolder} 

\begin{PRE}
 protected void doDeleteFolder(String foldername) throws CmsException
\end{PRE}

\begin{description}
\htmlDD Deletes a folder. \htmlBR
This is a very complex operation, because all sub-resources may be deleted
too. 

\begin{description}
\item {\bf Parameters:}  

foldername - the complete path of the folder.  
\item {\bf Throws:} CmsException  

if the folder couldn't be deleted, or if the user has not the rights to delete
this folder.  
\end{description}

\end{description}

o {\bf undeleteResource} 

\begin{PRE}
 public void undeleteResource(String filename) throws CmsException
\end{PRE}

\begin{description}
\htmlDD Undeletes a resource. 

\begin{description}
\item {\bf Parameters:}  

filename - the complete path of the file.  
\item {\bf Throws:} CmsException  

if the file couldn't be undeleted, or if the user has not the appropriate
rights to undelete the file.  
\end{description}

\end{description}

o {\bf doUndeleteFile} 

\begin{PRE}
 protected void doUndeleteFile(String filename) throws CmsException
\end{PRE}

\begin{description}
\htmlDD Undeletes a file. 

\begin{description}
\item {\bf Parameters:}  

filename - the complete path of the file.  
\item {\bf Throws:} CmsException  

if the file couldn't be undeleted, or if the user has not the appropriate
rights to undelete the file.  
\end{description}

\end{description}

o {\bf doUndeleteFolder} 

\begin{PRE}
 protected void doUndeleteFolder(String foldername) throws CmsException
\end{PRE}

\begin{description}
\htmlDD Undeletes a folder. \htmlBR
This is a very complex operation, because all sub-resources may be undeleted
too. 

\begin{description}
\item {\bf Parameters:}  

foldername - the complete path of the folder.  
\item {\bf Throws:} CmsException  

if the folder couldn't be undeleted, or if the user has not the rights to
undelete this folder.  
\end{description}

\end{description}

o {\bf deleteGroup} 

\begin{PRE}
 public void deleteGroup(String delgroup) throws CmsException
\end{PRE}

\begin{description}
\htmlDD Deletes a group. 

{\bf Security:} Only the admin user is allowed to delete a group. 

\begin{description}
\item {\bf Parameters:}  

delgroup - the name of the group.  
\item {\bf Throws:} CmsException  

if operation was not successful.  
\end{description}

\end{description}

o {\bf deleteProject} 

\begin{PRE}
 public void deleteProject(int id) throws CmsException
\end{PRE}

\begin{description}
\htmlDD Deletes a project. 

\begin{description}
\item {\bf Parameters:}  

id - the id of the project.  
\item {\bf Throws:} CmsException  

if operation was not successful.  
\end{description}

\end{description}

o {\bf deleteProperty} 

\begin{PRE}
 public void deleteProperty(String resourcename,
                            String property) throws CmsException
\end{PRE}

\begin{description}
\htmlDD Deletes a property for a file or folder. 

\begin{description}
\item {\bf Parameters:}  

resourcename - the name of a resource for which the property should be
deleted.  

property - the name of the property.  
\item {\bf Throws:} CmsException  

Throws if operation was not successful.  
\end{description}

\end{description}

o {\bf deletePropertydefinition} 

\begin{PRE}
 public void deletePropertydefinition(String name,
                                      String resourcetype) throws CmsException
\end{PRE}

\begin{description}
\htmlDD Deletes the property-definition for a resource type. 

\begin{description}
\item {\bf Parameters:}  

name - the name of the property-definition to delete.  

resourcetype - the name of the resource-type for the property-definition.  
\item {\bf Throws:} CmsException  

if operation was not successful.  
\end{description}

\end{description}

o {\bf deleteUser} 

\begin{PRE}
 public void deleteUser(int userId) throws CmsException
\end{PRE}

\begin{description}
\htmlDD Deletes a user from the Cms. 

{\bf Security:} Only a admin user is allowed to delete a user. 

\begin{description}
\item {\bf Parameters:}  

name - the Id of the user to be deleted.  
\item {\bf Throws:} CmsException  

if operation was not successful.  
\end{description}

\end{description}

o {\bf deleteUser} 

\begin{PRE}
 public void deleteUser(String username) throws CmsException
\end{PRE}

\begin{description}
\htmlDD Deletes a user from the Cms. 

{\bf Security:} Only a admin user is allowed to delete a user. 

\begin{description}
\item {\bf Parameters:}  

name - the name of the user to be deleted.  
\item {\bf Throws:} CmsException  

if operation was not successful.  
\end{description}

\end{description}

o {\bf deleteWebUser} 

\begin{PRE}
 public void deleteWebUser(int userId) throws CmsException
\end{PRE}

\begin{description}
\htmlDD Deletes a web user from the Cms. 

\begin{description}
\item {\bf Parameters:}  

name - the id of the user to be deleted.  
\item {\bf Throws:} CmsException  

if operation was not successful.  
\end{description}

\end{description}

o {\bf destroy} 

\begin{PRE}
 public void destroy() throws CmsException
\end{PRE}

\begin{description}
\htmlDD Destroys the resource borker and required modules and connections. 

\begin{description}
\item {\bf Throws:} CmsException  

if operation was not successful.  
\end{description}

\end{description}

o {\bf endTask} 

\begin{PRE}
 public void endTask(int taskid) throws CmsException
\end{PRE}

\begin{description}
\htmlDD Ends a task of the Cms. 

\begin{description}
\item {\bf Parameters:}  

taskid - the ID of the task to end.  
\item {\bf Throws:} CmsException  

if operation was not successful.  
\end{description}

\end{description}

o {\bf exportResources} 

\begin{PRE}
 public void exportResources(String exportFile,
                             String exportPaths[]) throws CmsException
\end{PRE}

\begin{description}
\htmlDD Exports cms-resources to a zip-file. 

\begin{description}
\item {\bf Parameters:}  

exportFile - the name (absolute Path) of the export resource (zip-file).  

exportPath - the name (absolute Path) of folder from which should be exported.
 
\item {\bf Throws:} CmsException  

if operation was not successful.  
\end{description}

\end{description}

o {\bf exportResources} 

\begin{PRE}
 public void exportResources(String exportFile,
                             String exportPaths[],
                             boolean includeSystem,
                             boolean excludeUnchanged) throws CmsException
\end{PRE}

\begin{description}
\htmlDD Exports cms-resources to a zip-file. 

\begin{description}
\item {\bf Parameters:}  

exportFile - the name (absolute Path) of the export resource (zip-file).  

exportPath - the name (absolute Path) of folder from which should be exported.
 

includeSystem - indicates if the system resources will be included in the
export.  

excludeUnchanged - {\tt true}, if unchanged files should be excluded.  
\item {\bf Throws:} CmsException  

if operation was not successful.  
\end{description}

\end{description}

o {\bf exportResources} 

\begin{PRE}
 public void exportResources(String exportFile,
                             String exportPaths[],
                             boolean includeSystem,
                             boolean excludeUnchanged,
                             boolean exportUserdata) throws CmsException
\end{PRE}

\begin{description}
\htmlDD Exports cms-resources to a zip-file. 

\begin{description}
\item {\bf Parameters:}  

exportFile - the name (absolute Path) of the export resource (zip-file).  

exportPath - the name (absolute Path) of folder from which should be exported.
 

includeSystem - indicates if the system resources will be included in the
export.  

excludeUnchanged - {\tt true}, if unchanged files should be excluded.  
\item {\bf Throws:} CmsException  

if operation was not successful.  
\end{description}

\end{description}

o {\bf exportResource} 

\begin{PRE}
 public CmsFile exportResource(CmsFile file) throws CmsException
\end{PRE}

\begin{description}
\htmlDD Exports a resource. 

\end{description}

o {\bf exportModuledata} 

\begin{PRE}
 public void exportModuledata(String exportFile,
                              String exportChannels[],
                              String exportModules[]) throws CmsException
\end{PRE}

\begin{description}
\htmlDD Exports channels and moduledata to zip. {\bf Security:} only
Administrators can do this; 

\begin{description}
\item {\bf Parameters:}  

currentUser - user who requestd themethod  

currentProject - current project of the user  

exportFile - the name (absolute Path) of the export resource (zip)  

exportChannels - the names (absolute Path) of channels from which should be
exported  

exportModules - the names of modules from which should be exported  

cms - the cms-object to use for the export.  
\item {\bf Throws:} Throws  

CmsException if something goes wrong.  
\end{description}

\end{description}

o {\bf exportStaticResources} 

\begin{PRE}
 public void exportStaticResources(Vector startpoints,
                                   Vector projectResources,
                                   CmsPublishedResources changedResources) throws CmsException
\end{PRE}

\begin{description}
\htmlDD Creates a static export in the filesystem 

\begin{description}
\item {\bf Parameters:}  

startpoints - the startpoints for the export.  
\item {\bf Throws:} CmsException  

if operation was not successful.  
\end{description}

\end{description}

o {\bf getCmsObjectForStaticExport} 

\begin{PRE}
 public CmsObject getCmsObjectForStaticExport(CmsExportRequest dReq,
                                              CmsExportResponse dRes) throws CmsException
\end{PRE}

\begin{description}
\htmlDD Creates a special CmsObject for the static export. 

\begin{description}
\item {\bf Throws:} CmsException  

if operation was not successful.  
\end{description}

\end{description}

o {\bf forwardTask} 

\begin{PRE}
 public void forwardTask(int taskid,
                         String newRoleName,
                         String newUserName) throws CmsException
\end{PRE}

\begin{description}
\htmlDD Forwards a task to a new user. 

\begin{description}
\item {\bf Parameters:}  

taskid - the id of the task which will be forwarded.  

newRole - the new group for the task.  

newUser - the new user who gets the task.  
\item {\bf Throws:} CmsException  

if operation was not successful.  
\end{description}

\end{description}

o {\bf getAllAccessibleProjects} 

\begin{PRE}
 public Vector getAllAccessibleProjects() throws CmsException
\end{PRE}

\begin{description}
\htmlDD Returns all projects, which the current user can access. 

\begin{description}
\item {\bf Returns:}  

a Vector of objects of type {\tt CmsProject}.  
\item {\bf Throws:} CmsException  

if operation was not successful.  
\end{description}

\end{description}

o {\bf getAllManageableProjects} 

\begin{PRE}
 public Vector getAllManageableProjects() throws CmsException
\end{PRE}

\begin{description}
\htmlDD Returns all projects which are owned by the current user or which are
manageable for the group of the user. 

\begin{description}
\item {\bf Returns:}  

a Vector of objects of type {\tt CmsProject}.  
\item {\bf Throws:} CmsException  

if operation was not successful.  
\end{description}

\end{description}

o {\bf getAllBackupProjects} 

\begin{PRE}
 public Vector getAllBackupProjects() throws CmsException
\end{PRE}

\begin{description}
\htmlDD Returns a Vector with all projects from history 

\begin{description}
\item {\bf Returns:}  

Vector with all projects from history.  
\item {\bf Throws:} CmsException  

Throws CmsException if operation was not succesful.  
\end{description}

\end{description}

o {\bf getAllResourceTypes} 

\begin{PRE}
 public Hashtable getAllResourceTypes() throws CmsException
\end{PRE}

\begin{description}
\htmlDD Returns a Hashtable with all I\_CmsResourceTypes. 

\begin{description}
\item {\bf Throws:} CmsException  

if operation was not successful.  
\end{description}

\end{description}

o {\bf getCacheInfo} 

\begin{PRE}
 public Hashtable getCacheInfo()
\end{PRE}

\begin{description}
\htmlDD Gets information about the cache size. \htmlBR
The size of the following caching areas is returned: 

\begin{itemize}
\item GroupCache  
\item UserGroupCache  
\item ResourceCache  
\item SubResourceCache  
\item ProjectCache  
\item PropertyCache  
\item PropertyDefinitionCache  
\item PropertyDefinitionVectorCache 
\end{itemize}

\end{description}

o {\bf getChild} 

\begin{PRE}
 public Vector getChild(String groupname) throws CmsException
\end{PRE}

\begin{description}
\htmlDD Returns all child groups of a group. 

\begin{description}
\item {\bf Parameters:}  

groupname - the name of the group.  
\item {\bf Returns:}  

groups a Vector of all child groups or null.  
\item {\bf Throws:} CmsException  

if operation was not successful.  
\end{description}

\end{description}

o {\bf getChilds} 

\begin{PRE}
 public Vector getChilds(String groupname) throws CmsException
\end{PRE}

\begin{description}
\htmlDD Returns all child groups of a group. \htmlBR
This method also returns all sub-child groups of the current group. 

\begin{description}
\item {\bf Parameters:}  

groupname - the name of the group.  
\item {\bf Returns:}  

groups a Vector of all child groups or null.  
\item {\bf Throws:} CmsException  

if operation was not successful.  
\end{description}

\end{description}

o {\bf getConfigurations} 

\begin{PRE}
 public Configurations getConfigurations()
\end{PRE}

\begin{description}
\htmlDD Gets the configurations of the properties-file. 

\begin{description}
\item {\bf Returns:}  

the configurations of the properties-file.  
\end{description}

\end{description}

o {\bf getDirectGroupsOfUser} 

\begin{PRE}
 public Vector getDirectGroupsOfUser(String username) throws CmsException
\end{PRE}

\begin{description}
\htmlDD Gets all groups to which a given user directly belongs. 

\begin{description}
\item {\bf Parameters:}  

username - the name of the user to get all groups for.  
\item {\bf Returns:}  

a Vector of all groups of a user.  
\item {\bf Throws:} CmsException  

if operation was not successful.  
\end{description}

\end{description}

o {\bf getFilesInFolder} 

\begin{PRE}
 public Vector getFilesInFolder(String foldername) throws CmsException
\end{PRE}

\begin{description}
\htmlDD Returns a Vector with all files of a given folder. (only the direct
subfiles, not the files in subfolders) \htmlBR
Files of a folder can be read from an offline Project and the online Project. 

\begin{description}
\item {\bf Parameters:}  

foldername - the complete path to the folder.  
\item {\bf Returns:}  

subfiles a Vector with all files of the given folder.  
\item {\bf Throws:} CmsException  

if the user has not hte appropriate rigths to access or read the resource.  
\end{description}

\end{description}

o {\bf getFilesInFolder} 

\begin{PRE}
 public Vector getFilesInFolder(String foldername,
                                boolean includeDeleted) throws CmsException
\end{PRE}

\begin{description}
\htmlDD Returns a Vector with all files of a given folder. \htmlBR
Files of a folder can be read from an offline Project and the online Project. 

\begin{description}
\item {\bf Parameters:}  

foldername - the complete path to the folder.  

includeDeleted - Include if the folder is marked as deleted  
\item {\bf Returns:}  

subfiles a Vector with all files of the given folder.  
\item {\bf Throws:} CmsException  

if the user has not hte appropriate rigths to access or read the resource.  
\end{description}

\end{description}

o {\bf getFilesWithProperty} 

\begin{PRE}
 public Vector getFilesWithProperty(String propertyDefinition,
                                    String propertyValue) throws CmsException
\end{PRE}

\begin{description}
\htmlDD Returns a Vector with all resource-names of the resources that have
set the given property to the given value. 

\begin{description}
\item {\bf Parameters:}  

propertydef - the name of the property-definition to check.  

property - the value of the property for the resource.  
\item {\bf Returns:}  

a Vector with all names of the resources.  
\item {\bf Throws:} CmsException  

if operation was not successful.  
\end{description}

\end{description}

o {\bf getFileSystemChanges} 

\begin{PRE}
 public long getFileSystemChanges()
\end{PRE}

\begin{description}
\htmlDD This method can be called, to determine if the file-system was changed
in the past. \htmlBR
A module can compare its previously stored number with the returned number. If
they differ, the file system has been changed. 

\begin{description}
\item {\bf Returns:}  

the number of file-system-changes.  
\end{description}

\end{description}

o {\bf getFileSystemFolderChanges} 

\begin{PRE}
 public long getFileSystemFolderChanges()
\end{PRE}

\begin{description}
\htmlDD This method can be called, to determine if the file-system was changed
in the past. \htmlBR
A module can compare its previously stored number with the returned number. If
they differ, the file system has been changed. 

\begin{description}
\item {\bf Returns:}  

the number of file-system-changes.  
\end{description}

\end{description}

o {\bf getFolderTree} 

\begin{PRE}
 public Vector getFolderTree() throws CmsException
\end{PRE}

\begin{description}
\htmlDD Returns a Vector with the complete folder-tree for this
project.\htmlBR
Subfolders can be read from an offline project and the online project. \htmlBR

\begin{description}
\item {\bf Returns:}  

subfolders A Vector with the complete folder-tree for this project.  
\item {\bf Throws:} CmsException  

Throws CmsException if operation was not succesful.  
\end{description}

\end{description}

o {\bf getGroups} 

\begin{PRE}
 public Vector getGroups() throws CmsException
\end{PRE}

\begin{description}
\htmlDD Returns all groups in the Cms. 

\begin{description}
\item {\bf Returns:}  

a Vector of all groups in the Cms.  
\item {\bf Throws:} CmsException  

if operation was not successful  
\end{description}

\end{description}

o {\bf getGroupsOfUser} 

\begin{PRE}
 public Vector getGroupsOfUser(String username) throws CmsException
\end{PRE}

\begin{description}
\htmlDD Gets all groups of a user. 

\begin{description}
\item {\bf Parameters:}  

username - the name of the user to get all groups for.  
\item {\bf Returns:}  

Vector of all groups of a user.  
\item {\bf Throws:} CmsException  

if operation was not succesful.  
\end{description}

\end{description}

o {\bf getLauncherManager} 

\begin{PRE}
 public CmsLauncherManager getLauncherManager()
\end{PRE}

\begin{description}
\htmlDD Get the launcher manager used with this instance of CmsObject.
Creation date: (10/23/00 14:50:15) 

\begin{description}
\item {\bf Returns:}  

com.opencms.launcher.CmsLauncherManager  
\end{description}

\end{description}

o {\bf getOnlineElementCache} 

\begin{PRE}
 public CmsElementCache getOnlineElementCache()
\end{PRE}

\begin{description}
\htmlDD Gets the ElementCache used for the online project. 

\begin{description}
\item {\bf Returns:}  

CmsElementCache  
\end{description}

\end{description}

o {\bf getLinkRules} 

\begin{PRE}
 public String[] getLinkRules(int state)
\end{PRE}

\begin{description}
\htmlDD Returns the ruleset for link replacement. 

\begin{description}
\item {\bf Parameters:}  

state. - defines which set is needed.  
\item {\bf Returns:}  

String[] the ruleset.  
\end{description}

\end{description}

o {\bf getLinkSubstitution} 

\begin{PRE}
 public String getLinkSubstitution(String link)
\end{PRE}

\begin{description}
\htmlDD Replaces the link according to the rules and registers it to the
requestcontex if we are in export modus. 

\begin{description}
\item {\bf Parameters:}  

link. - The link to process.  
\item {\bf Returns:}  

String The substituded link.  
\end{description}

\end{description}

o {\bf getStaticExportStartPoints} 

\begin{PRE}
 public Vector getStaticExportStartPoints()
\end{PRE}

\begin{description}
\htmlDD Returns a Vector (of Strings) with the names of the vfs resources
(files and folders) where the export should start. 

\begin{description}
\item {\bf Returns:}  

Vector with resources for the export.  
\end{description}

\end{description}

o {\bf getStaticExportPath} 

\begin{PRE}
 public String getStaticExportPath()
\end{PRE}

\begin{description}
\htmlDD Returns the exportpath for the static export. 

\end{description}

o {\bf isStaticExportEnabled} 

\begin{PRE}
 public boolean isStaticExportEnabled()
\end{PRE}

\begin{description}
\htmlDD Returns true, if the static export is enabled. 

\begin{description}
\item {\bf Returns:}  

true, if the static export is enabled.  
\end{description}

\end{description}

o {\bf getUrlPrefixArray} 

\begin{PRE}
 public String[] getUrlPrefixArray()
\end{PRE}

\begin{description}
\htmlDD Gets the prefix array for the linkreplacement 

\begin{description}
\item {\bf Returns:}  

String[4]  
\end{description}

\end{description}

o {\bf getMode} 

\begin{PRE}
 public int getMode()
\end{PRE}

\begin{description}
\htmlDD Returns the mode this cmsObject is runnig in. AUTO mode (-1) means it
is no special case and returns online ore offline depending on the current
project. 

\begin{description}
\item {\bf Returns:}  

int The modus of this cmsObject.  
\end{description}

\end{description}

o {\bf setMode} 

\begin{PRE}
 public void setMode(int mode)
\end{PRE}

\begin{description}
\htmlDD Sets the mode this CmsObject runs in. Used for static export. 

\begin{description}
\item {\bf Parameters:}  

e - mode.  
\end{description}

\end{description}

o {\bf getVariantDependencies} 

\begin{PRE}
 public Hashtable getVariantDependencies()
\end{PRE}

\begin{description}
\htmlDD Gets the hashtable with the variant dependencies used for the
elementcache. 

\begin{description}
\item {\bf Returns:}  

Hashtable  
\end{description}

\end{description}

o {\bf getReadingpermittedGroup} 

\begin{PRE}
 public String getReadingpermittedGroup(int projectId,
                                        String resource) throws CmsException
\end{PRE}

\begin{description}
\htmlDD Checks which Group can read the resource and all the parent folders. 

\begin{description}
\item {\bf Parameters:}  

projectid - the project to check the permission.  

res - The resource name to be checked.  
\item {\bf Returns:}  

The Group Id of the Group which can read the resource.  null for all Groups
and  Admingroup for no Group.  
\end{description}

\end{description}

o {\bf getParent} 

\begin{PRE}
 public CmsGroup getParent(String groupname) throws CmsException
\end{PRE}

\begin{description}
\htmlDD Returns the parent group of a group. 

\begin{description}
\item {\bf Parameters:}  

groupname - the name of the group.  
\item {\bf Returns:}  

group the parent group or null.  
\item {\bf Throws:} CmsException  

if operation was not successful.  
\end{description}

\end{description}

o {\bf getRegistry} 

\begin{PRE}
 public I\_CmsRegistry getRegistry() throws CmsException
\end{PRE}

\begin{description}
\htmlDD Gets the Registry. 

\begin{description}
\item {\bf Throws:} CmsException  

if access is not allowed.  
\end{description}

\end{description}

o {\bf getRequestContext} 

\begin{PRE}
 public CmsRequestContext getRequestContext()
\end{PRE}

\begin{description}
\htmlDD Returns the current request-context. 

\begin{description}
\item {\bf Returns:}  

the current request-context.  
\end{description}

\end{description}

o {\bf getResourcesInFolder} 

\begin{PRE}
 public Vector getResourcesInFolder(String folder) throws CmsException
\end{PRE}

\begin{description}
\htmlDD Returns a Vector with the subresources for a folder.\htmlBR

\begin{description}
\item {\bf Parameters:}  

folder - The name of the folder to get the subresources from.  
\item {\bf Returns:}  

subfolders A Vector with resources.  
\item {\bf Throws:} CmsException  

Throws CmsException if operation was not succesful.  
\end{description}

\end{description}

o {\bf getResourcesWithProperty} 

\begin{PRE}
 public Vector getResourcesWithProperty(String propertyDefinition,
                                        String propertyValue,
                                        int resourceType) throws CmsException
\end{PRE}

\begin{description}
\htmlDD Returns a Vector with all resources of the given type that have set
the given property to the given value. {\bf Security:} All users are granted. 

\begin{description}
\item {\bf Parameters:}  

propertyDefinition, - the name of the propertydefinition to check.  

propertyValue, - the value of the property for the resource.  

resourceType - The resource type of the resource  
\item {\bf Returns:}  

Vector with all resources.  
\item {\bf Throws:} CmsException  

Throws CmsException if operation was not succesful.  
\end{description}

\end{description}

o {\bf getResourcesWithProperty} 

\begin{PRE}
 public Vector getResourcesWithProperty(String propertyDefinition) throws CmsException
\end{PRE}

\begin{description}
\htmlDD Returns a Vector with all resources of the given type that have set
the given property. {\bf Security:} All users are granted. 

\begin{description}
\item {\bf Parameters:}  

propertyDefinition, - the name of the propertydefinition to check.  

propertyValue, - the value of the property for the resource.  

resourceType - The resource type of the resource  
\item {\bf Returns:}  

Vector with all resources.  
\item {\bf Throws:} CmsException  

Throws CmsException if operation was not succesful.  
\end{description}

\end{description}

o {\bf getResourceType} 

\begin{PRE}
 public I\_CmsResourceType getResourceType(int resourceType) throws CmsException
\end{PRE}

\begin{description}
\htmlDD Returns a I\_CmsResourceType. 

\begin{description}
\item {\bf Parameters:}  

resourceType - the id of the resource to get.  
\item {\bf Returns:}  

a CmsResourceType.  
\item {\bf Throws:} CmsException  

if operation was not successful.  
\end{description}

\end{description}

o {\bf getResourceType} 

\begin{PRE}
 public I\_CmsResourceType getResourceType(String resourceType) throws CmsException
\end{PRE}

\begin{description}
\htmlDD Returns a I\_CmsResourceType. 

\begin{description}
\item {\bf Parameters:}  

resourceType - the name of the resource to get.  
\item {\bf Returns:}  

a CmsResourceType.  
\item {\bf Throws:} CmsException  

if operation was not successful.  
\end{description}

\end{description}

o {\bf getSubFolders} 

\begin{PRE}
 public Vector getSubFolders(String foldername) throws CmsException
\end{PRE}

\begin{description}
\htmlDD Returns a Vector with all subfolders of a given folder. 

\begin{description}
\item {\bf Parameters:}  

foldername - the complete path to the folder.  
\item {\bf Returns:}  

subfolders a Vector with all subfolders for the given folder.  
\item {\bf Throws:} CmsException  

if the user has not the rights to access or read the resource.  
\end{description}

\end{description}

o {\bf getSubFolders} 

\begin{PRE}
 public Vector getSubFolders(String foldername,
                             boolean includeDeleted) throws CmsException
\end{PRE}

\begin{description}
\htmlDD Returns a Vector with all subfolders of a given folder. 

\begin{description}
\item {\bf Parameters:}  

foldername - the complete path to the folder.  

includeDeleted - Include if the folder is marked as deleted  
\item {\bf Returns:}  

subfolders a Vector with all subfolders (CmsFolder Objects) for the given
folder.  
\item {\bf Throws:} CmsException  

if the user has not the rights to access or read the resource.  
\end{description}

\end{description}

o {\bf getTaskPar} 

\begin{PRE}
 public String getTaskPar(int taskid,
                          String parname) throws CmsException
\end{PRE}

\begin{description}
\htmlDD Get a parameter value for a task. 

\begin{description}
\item {\bf Parameters:}  

taskid - the id of the task.  

parname - the name of the parameter.  
\item {\bf Returns:}  

the parameter value.  
\item {\bf Throws:} CmsException  

if operation was not successful.  
\end{description}

\end{description}

o {\bf getTaskType} 

\begin{PRE}
 public int getTaskType(String taskname) throws CmsException
\end{PRE}

\begin{description}
\htmlDD Get the template task id fo a given taskname. 

\begin{description}
\item {\bf Parameters:}  

taskname - the name of the task.  
\item {\bf Returns:}  

the id of the task template.  
\item {\bf Throws:} CmsException  

if operation was not successful.  
\end{description}

\end{description}

o {\bf getUsers} 

\begin{PRE}
 public Vector getUsers() throws CmsException
\end{PRE}

\begin{description}
\htmlDD Returns all users in the Cms. 

\begin{description}
\item {\bf Returns:}  

a Vector of all users in the Cms.  
\item {\bf Throws:} CmsException  

if operation was not successful.  
\end{description}

\end{description}

o {\bf getUsers} 

\begin{PRE}
 public Vector getUsers(int type) throws CmsException
\end{PRE}

\begin{description}
\htmlDD Returns all users of the given type in the Cms. 

\begin{description}
\item {\bf Parameters:}  

type - the type of the users.  
\item {\bf Returns:}  

vector of all users of the given type in the Cms.  
\item {\bf Throws:} CmsException  

if operation was not successful.  
\end{description}

\end{description}

o {\bf getUsers} 

\begin{PRE}
 public Vector getUsers(int type,
                        String namefilter) throws CmsException
\end{PRE}

\begin{description}
\htmlDD Returns all users from a given type that start with a specified string


\begin{description}
\item {\bf Parameters:}  

type - the type of the users.  

namestart - The filter for the username  
\item {\bf Returns:}  

vector of all users of the given type in the Cms.  
\item {\bf Throws:} CmsException  

if operation was not successful.  
\end{description}

\end{description}

o {\bf getUsersOfGroup} 

\begin{PRE}
 public Vector getUsersOfGroup(String groupname) throws CmsException
\end{PRE}

\begin{description}
\htmlDD Gets all users of a group. 

\begin{description}
\item {\bf Parameters:}  

groupname - the name of the group to get all users for.  
\item {\bf Returns:}  

all users in the group.  
\item {\bf Throws:} CmsException  

if operation was not successful.  
\end{description}

\end{description}

o {\bf getUsersByLastname} 

\begin{PRE}
 public Vector getUsersByLastname(String Lastname,
                                  int UserType,
                                  int UserStatus,
                                  int wasLoggedIn,
                                  int nMax) throws CmsException
\end{PRE}

\begin{description}
\htmlDD Gets all users with a certain Lastname. 

\begin{description}
\item {\bf Parameters:}  

Lastname - the start of the users lastname  

UserType - webuser or systemuser  

UserStatus - enabled, disabled  

wasLoggedIn - was the user ever locked in?  

nMax - max number of results  
\item {\bf Returns:}  

the users.  
\item {\bf Throws:} CmsException  

if operation was not successful.  
\end{description}

\end{description}

o {\bf importFolder} 

\begin{PRE}
 public void importFolder(String importFile,
                          String importPath) throws CmsException
\end{PRE}

\begin{description}
\htmlDD Imports a import-resource (folder or zipfile) to the cms. 

\begin{description}
\item {\bf Parameters:}  

importFile - the name (absolute Path) of the import resource (zipfile or
folder).  

importPath - the name (absolute Path) of the folder in which should be
imported.  
\item {\bf Throws:} CmsException  

if operation was not successful.  
\end{description}

\end{description}

o {\bf importResource} 

\begin{PRE}
 public CmsResource importResource(String source,
                                   String destination,
                                   String type,
                                   String user,
                                   String group,
                                   String access,
                                   Hashtable properties,
                                   String launcherStartClass,
                                   byte content[],
                                   String importPath) throws CmsException
\end{PRE}

\begin{description}
\htmlDD Imports a resource to the cms. 

\begin{description}
\item {\bf Parameters:}  

source - the name of the import resource (zipfile or folder).  

destination - the name (absolute Path) of the folder in which should be
imported.  

type - the type of the resource  

user - the owner of the resource  

group - the group of the resource  

access - the access flags of the resource  

properties - the properties of the resource  

launcherStartClass - the name of launcher start class  

content - the content of the resource  

importPath - the name of the import path  
\item {\bf Throws:} CmsException  

if operation was not successful.  
\end{description}

\end{description}

o {\bf importResources} 

\begin{PRE}
 public void importResources(String importFile,
                             String importPath) throws CmsException
\end{PRE}

\begin{description}
\htmlDD Imports a import-resource (folder or zip-file) to the cms. 

\begin{description}
\item {\bf Parameters:}  

importFile - the name (absolute Path) of the import resource (zipfile or
folder).  

importPath - the name (absolute Path) of folder in which should be imported.  
\item {\bf Throws:} CmsException  

if operation was not successful.  
\end{description}

\end{description}

o {\bf init} 

\begin{PRE}
 public void init(I\_CmsResourceBroker broker) throws CmsException
\end{PRE}

\begin{description}
\htmlDD Initializes the CmsObject without a request-context (current-user,
current-group, current-project). 

\begin{description}
\item {\bf Parameters:}  

broker - the resourcebroker to access the database.  
\item {\bf Throws:} CmsException  

if operation was not successful.  
\end{description}

\end{description}

o {\bf init} 

\begin{PRE}
 public void init(I\_CmsResourceBroker broker,
                  I\_CmsRequest req,
                  I\_CmsResponse resp,
                  String user,
                  String currentGroup,
                  int currentProjectId,
                  boolean streaming,
                  CmsElementCache elementCache,
                  CmsCoreSession sessionStorage) throws CmsException
\end{PRE}

\begin{description}
\htmlDD Initializes the CmsObject for each request. 

\begin{description}
\item {\bf Parameters:}  

broker - the resourcebroker to access the database.  

req - the CmsRequest.  

resp - the CmsResponse.  

user - the current user for this request.  

currentGroup - the current group for this request.  

currentProjectId - the current projectId for this request.  

streaming - {\tt true} if streaming should be enabled while creating the
request context, {\tt false} otherwise.  

elementCache - Starting point for the element cache or {\tt null} if the
element cache should be disabled.  
\item {\bf Throws:} CmsException  

if operation was not successful.  
\end{description}

\end{description}

o {\bf isAdmin} 

\begin{PRE}
 public boolean isAdmin() throws CmsException
\end{PRE}

\begin{description}
\htmlDD Checks, if the users current group is the admin-group. 

\begin{description}
\item {\bf Returns:}  

{\tt true}, if the users current group is the admin-group; {\tt false}
otherwise.  
\item {\bf Throws:} CmsException  

if operation was not successful.  
\end{description}

\end{description}

o {\bf isManagerOfProject} 

\begin{PRE}
 public boolean isManagerOfProject() throws CmsException
\end{PRE}

\begin{description}
\htmlDD Checks, if the user has management access to the project. 

\begin{description}
\item {\bf Returns:}  

{\tt true}, if the users current group is the admin-group; {\tt false}
otherwise.  
\item {\bf Throws:} CmsException  

if operation was not successful.  
\end{description}

\end{description}

o {\bf lockedBy} 

\begin{PRE}
 public CmsUser lockedBy(CmsResource resource) throws CmsException
\end{PRE}

\begin{description}
\htmlDD Returns the user, who has locked a given resource. \htmlBR
A user can lock a resource, so he is the only one who can write this resource.
This methods checks, who has locked a resource. 

\begin{description}
\item {\bf Parameters:}  

resource - the resource to check.  
\item {\bf Returns:}  

the user who has locked the resource.  
\item {\bf Throws:} CmsException  

if operation was not successful.  
\end{description}

\end{description}

o {\bf lockedBy} 

\begin{PRE}
 public CmsUser lockedBy(String resource) throws CmsException
\end{PRE}

\begin{description}
\htmlDD Returns the user, who has locked a given resource. \htmlBR
A user can lock a resource, so he is the only one who can write this resource.
This methods checks, who has locked a resource. 

\begin{description}
\item {\bf Parameters:}  

resource - The complete path to the resource.  
\item {\bf Returns:}  

the user who has locked a resource.  
\item {\bf Throws:} CmsException  

if operation was not successful.  
\end{description}

\end{description}

o {\bf lockResource} 

\begin{PRE}
 public void lockResource(String resource) throws CmsException
\end{PRE}

\begin{description}
\htmlDD Locks the given resource. \htmlBR
A user can lock a resource, so he is the only one who can write this resource.


\begin{description}
\item {\bf Parameters:}  

resource - The complete path to the resource to lock.  
\item {\bf Throws:} CmsException  

if the user has not the rights to lock this resource. It will also be thrown,
if there is an existing lock.  
\end{description}

\end{description}

o {\bf lockResource} 

\begin{PRE}
 public void lockResource(String resource,
                          boolean force) throws CmsException
\end{PRE}

\begin{description}
\htmlDD Locks a given resource. \htmlBR
A user can lock a resource, so he is the only one who can write this resource.


\begin{description}
\item {\bf Parameters:}  

resource - the complete path to the resource to lock.  

force - if force is {\tt true}, a existing locking will be overwritten.  
\item {\bf Throws:} CmsException  

if the user has not the rights to lock this resource. It will also be thrown,
if there is a existing lock and force was set to false.  
\end{description}

\end{description}

o {\bf doLockResource} 

\begin{PRE}
 protected void doLockResource(String resource,
                               boolean force) throws CmsException
\end{PRE}

\begin{description}
\htmlDD Locks a given resource. \htmlBR
A user can lock a resource, so he is the only one who can write this resource.


\begin{description}
\item {\bf Parameters:}  

resource - the complete path to the resource to lock.  

force - if force is {\tt true}, a existing locking will be overwritten.  
\item {\bf Throws:} CmsException  

if the user has not the rights to lock this resource. It will also be thrown,
if there is a existing lock and force was set to false.  
\end{description}

\end{description}

o {\bf loginUser} 

\begin{PRE}
 public String loginUser(String username,
                         String password) throws CmsException
\end{PRE}

\begin{description}
\htmlDD Logs a user into the Cms, if the password is correct. 

\begin{description}
\item {\bf Parameters:}  

username - the name of the user.  

password - the password of the user.  
\item {\bf Returns:}  

the name of the logged in user.  
\item {\bf Throws:} CmsException  

if operation was not successful  
\end{description}

\end{description}

o {\bf loginWebUser} 

\begin{PRE}
 public String loginWebUser(String username,
                            String password) throws CmsException
\end{PRE}

\begin{description}
\htmlDD Logs a web user into the Cms, if the password is correct. 

\begin{description}
\item {\bf Parameters:}  

username - the name of the user.  

password - the password of the user.  
\item {\bf Returns:}  

the name of the logged in user.  
\item {\bf Throws:} CmsException  

if operation was not successful  
\end{description}

\end{description}

o {\bf moveFile} 

\begin{PRE}
 public void moveFile(String source,
                      String destination) throws CmsException
\end{PRE}

\begin{description}
\htmlDD {\bf Note: moveFile() is deprecated.} {\it Use moveResource instead.} 

Moves a file to the given destination. 

\begin{description}
\item {\bf Parameters:}  

source - the complete path of the sourcefile.  

destination - the complete path of the destinationfile.  
\item {\bf Throws:} CmsException  

if the user has not the rights to move this resource, or if the file couldn't
be moved.  
\end{description}

\end{description}

o {\bf moveResource} 

\begin{PRE}
 public void moveResource(String source,
                          String destination) throws CmsException
\end{PRE}

\begin{description}
\htmlDD Moves a resource to the given destination. 

\begin{description}
\item {\bf Parameters:}  

source - the complete path of the sourcefile.  

destination - the complete path of the destinationfile.  
\item {\bf Throws:} CmsException  

if the user has not the rights to move this resource, or if the file couldn't
be moved.  
\end{description}

\end{description}

o {\bf doMoveFile} 

\begin{PRE}
 protected void doMoveFile(String source,
                           String destination) throws CmsException
\end{PRE}

\begin{description}
\htmlDD Moves a file to the given destination. 

\begin{description}
\item {\bf Parameters:}  

source - the complete path of the sourcefile.  

destination - the complete path of the destinationfile.  
\item {\bf Throws:} CmsException  

if the user has not the rights to move this resource, or if the file couldn't
be moved.  
\end{description}

\end{description}

o {\bf onlineProject} 

\begin{PRE}
 public CmsProject onlineProject() throws CmsException
\end{PRE}

\begin{description}
\htmlDD Returns the online project. 

This is the default project. All anonymous (or guest) user will see the
resources of this project. 

\begin{description}
\item {\bf Returns:}  

the online project object.  
\item {\bf Throws:} CmsException  

if operation was not successful.  
\end{description}

\end{description}

o {\bf publishProject} 

\begin{PRE}
 public void publishProject(int id) throws CmsException
\end{PRE}

\begin{description}
\htmlDD Publishes a project. 

\begin{description}
\item {\bf Parameters:}  

id - the id of the project to be published.  
\item {\bf Returns:}  

a Vector of resources, that have been changed.  
\item {\bf Throws:} CmsException  

if operation was not successful.  
\end{description}

\end{description}

o {\bf publishResource} 

\begin{PRE}
 public void publishResource(String resourcename) throws CmsException
\end{PRE}

\begin{description}
\htmlDD Publishes a single resource. 

\begin{description}
\item {\bf Parameters:}  

id - the id of the project to be published.  
\item {\bf Returns:}  

a Vector of resources, that have been changed.  
\item {\bf Throws:} CmsException  

if operation was not successful.  
\end{description}

\end{description}

o {\bf readAgent} 

\begin{PRE}
 public CmsUser readAgent(CmsTask task) throws CmsException
\end{PRE}

\begin{description}
\htmlDD Reads the agent of a task from the OpenCms. 

\begin{description}
\item {\bf Parameters:}  

task - the task to read the agent from.  
\item {\bf Returns:}  

the owner of a task.  
\item {\bf Throws:} CmsException  

if operation was not successful.  
\end{description}

\end{description}

o {\bf readAllFileHeaders} 

\begin{PRE}
 public Vector readAllFileHeaders(String filename) throws CmsException
\end{PRE}

\begin{description}
\htmlDD {\bf Note: readAllFileHeaders() is deprecated.} {\it For reading the
file history use method readAllFileHeadersForHist} 

Reads all file headers of a file in the OpenCms. \htmlBR
This method returns a vector with all file headers, i.e. the file headers of a
file, independent of the project they were attached to.\htmlBR
The reading excludes the filecontent. 

\begin{description}
\item {\bf Parameters:}  

filename - the name of the file to be read.  
\item {\bf Returns:}  

a Vector of file headers read from the Cms.  
\item {\bf Throws:} CmsException  

if operation was not successful.  
\end{description}

\end{description}

o {\bf readAllFileHeadersForHist} 

\begin{PRE}
 public Vector readAllFileHeadersForHist(String filename) throws CmsException
\end{PRE}

\begin{description}
\htmlDD Reads all file headers of a file in the OpenCms. \htmlBR
This method returns a vector with the history of all file headers, i.e. the
file headers of a file, independent of the project they were attached
to.\htmlBR
The reading excludes the filecontent. 

\begin{description}
\item {\bf Parameters:}  

filename - the name of the file to be read.  
\item {\bf Returns:}  

a Vector of file headers read from the Cms.  
\item {\bf Throws:} CmsException  

if operation was not successful.  
\end{description}

\end{description}

o {\bf readAllProjectResources} 

\begin{PRE}
 public Vector readAllProjectResources(int projectId) throws CmsException
\end{PRE}

\begin{description}
\htmlDD select all projectResources from an given project 

\begin{description}
\item {\bf Parameters:}  

project - The project in which the resource is used.  
\item {\bf Throws:} CmsException  

Throws CmsException if operation was not succesful  
\end{description}

\end{description}

o {\bf readAllProperties} 

\begin{PRE}
 public Hashtable readAllProperties(String name) throws CmsException
\end{PRE}

\begin{description}
\htmlDD Returns a list of all properties of a file or folder. 

\begin{description}
\item {\bf Parameters:}  

name - the name of the resource for which the property has to be read.  
\item {\bf Returns:}  

a Vector of properties as Strings.  
\item {\bf Throws:} CmsException  

if operation was not succesful.  
\end{description}

\end{description}

o {\bf readAllPropertydefinitions} 

\begin{PRE}
 public Vector readAllPropertydefinitions(int id,
                                          int type) throws CmsException
\end{PRE}

\begin{description}
\htmlDD {\bf Note: readAllPropertydefinitions() is deprecated.} {\it Use the
method readAllPropertydefinitions without type of propertydefinition instead} 

Reads all property-definitions for the given resource type. 

\begin{description}
\item {\bf Parameters:}  

id - the id of the resource type to read the property-definitions for.  

type - the type of the property-definition (normal{\htmlBar}optional).  
\item {\bf Returns:}  

a Vector with property-defenitions for the resource type. The Vector may be
empty.  
\item {\bf Throws:} CmsException  

if operation was not successful.  
\end{description}

\end{description}

o {\bf readAllPropertydefinitions} 

\begin{PRE}
 public Vector readAllPropertydefinitions(int resourceType) throws CmsException
\end{PRE}

\begin{description}
\htmlDD Reads all property-definitions for the given resource type. 

\begin{description}
\item {\bf Parameters:}  

id - the id of the resource type to read the property-definitions for.  
\item {\bf Returns:}  

a Vector with property-defenitions for the resource type. The Vector may be
empty.  
\item {\bf Throws:} CmsException  

if operation was not successful.  
\end{description}

\end{description}

o {\bf readAllPropertydefinitions} 

\begin{PRE}
 public Vector readAllPropertydefinitions(String resourcetype) throws CmsException
\end{PRE}

\begin{description}
\htmlDD Reads all property-definitions for the given resource type. 

\begin{description}
\item {\bf Parameters:}  

resourcetype - the name of the resource type to read the property-definitions
for.  
\item {\bf Returns:}  

a Vector with property-defenitions for the resource type. The Vector may be
empty.  
\item {\bf Throws:} CmsException  

if operation was not successful.  
\end{description}

\end{description}

o {\bf readAllPropertydefinitions} 

\begin{PRE}
 public Vector readAllPropertydefinitions(String resourcetype,
                                          int type) throws CmsException
\end{PRE}

\begin{description}
\htmlDD {\bf Note: readAllPropertydefinitions() is deprecated.} {\it Use the
method readAllPropertydefinitions without type of propertydefinition instead} 

Reads all property-definitions for the given resource type. 

\begin{description}
\item {\bf Parameters:}  

resourcetype - The name of the resource type to read the property-definitions
for.  

type - the type of the property-definition (normal{\htmlBar}optional).  
\item {\bf Returns:}  

a Vector with property-defenitions for the resource type. The Vector may be
empty.  
\item {\bf Throws:} CmsException  

if operation was not successful.  
\end{description}

\end{description}

o {\bf readExportPath} 

\begin{PRE}
 public String readExportPath() throws CmsException
\end{PRE}

\begin{description}
\htmlDD Reads the export-path of the system. This path is used for db-export
and db-import. 

\begin{description}
\item {\bf Returns:}  

the exportpath.  
\item {\bf Throws:} CmsException  

if operation was not successful.  
\end{description}

\end{description}

o {\bf readExportLink} 

\begin{PRE}
 public CmsExportLink readExportLink(String request) throws CmsException
\end{PRE}

\begin{description}
\htmlDD Reads a exportrequest from the Cms. 

\begin{description}
\item {\bf Parameters:}  

request - the reourcename with the url parameter.  
\item {\bf Returns:}  

CmsExportLink the read exportrequest.  
\item {\bf Throws:} CmsException  

if the user has not the rights to read this resource, or if it couldn't be
read.  
\end{description}

\end{description}

o {\bf readExportLinkHeader} 

\begin{PRE}
 public CmsExportLink readExportLinkHeader(String request) throws CmsException
\end{PRE}

\begin{description}
\htmlDD Reads a exportrequest without the dependencies from the Cms. 

\begin{description}
\item {\bf Parameters:}  

request - the reourcename with the url parameter.  
\item {\bf Returns:}  

CmsExportLink the read exportrequest.  
\item {\bf Throws:} CmsException  

if the user has not the rights to read this resource, or if it couldn't be
read.  
\end{description}

\end{description}

o {\bf writeExportLink} 

\begin{PRE}
 public void writeExportLink(CmsExportLink link) throws CmsException
\end{PRE}

\begin{description}
\htmlDD Writes an exportlink to the Cms. 

\begin{description}
\item {\bf Parameters:}  

link - the cmsexportlink object to write.  
\item {\bf Throws:} CmsException  

if something goes wrong.  
\end{description}

\end{description}

o {\bf deleteExportLink} 

\begin{PRE}
 public void deleteExportLink(String link) throws CmsException
\end{PRE}

\begin{description}
\htmlDD Deletes an exportlink in the database. 

\begin{description}
\item {\bf Parameters:}  

link - the name of the link  
\end{description}

\end{description}

o {\bf deleteExportLink} 

\begin{PRE}
 public void deleteExportLink(CmsExportLink link) throws CmsException
\end{PRE}

\begin{description}
\htmlDD Deletes an exportlink in the database. 

\begin{description}
\item {\bf Parameters:}  

link - the cmsExportLink object to delete.  
\end{description}

\end{description}

o {\bf getDependingExportLinks} 

\begin{PRE}
 public Vector getDependingExportLinks(Vector res) throws CmsException
\end{PRE}

\begin{description}
\htmlDD Reads all export links that depend on the resource. 

\begin{description}
\item {\bf Parameters:}  

res. - The resourceName() of the resources that has changed (or the String 
that describes a contentdefinition).  
\item {\bf Returns:}  

a Vector(of Strings) with the linkrequest names.  
\end{description}

\end{description}

o {\bf writeExportLinkProcessedState} 

\begin{PRE}
 public void writeExportLinkProcessedState(CmsExportLink link) throws CmsException
\end{PRE}

\begin{description}
\htmlDD Sets one exportLink to procecced. 

\begin{description}
\item {\bf Parameters:}  

link - the cmsexportlink.  
\item {\bf Throws:} CmsException  

if something goes wrong.  
\end{description}

\end{description}

o {\bf readFile} 

\begin{PRE}
 public CmsFile readFile(String filename) throws CmsException
\end{PRE}

\begin{description}
\htmlDD Reads a file from the Cms. 

\begin{description}
\item {\bf Parameters:}  

filename - the complete path to the file.  
\item {\bf Returns:}  

file the read file.  
\item {\bf Throws:} CmsException  

if the user has not the rights to read this resource, or if the file couldn't
be read.  
\end{description}

\end{description}

o {\bf readFile} 

\begin{PRE}
 public CmsFile readFile(String filename,
                         boolean includeDeleted) throws CmsException
\end{PRE}

\begin{description}
\htmlDD Reads a file from the Cms. 

\begin{description}
\item {\bf Parameters:}  

filename - the complete path to the file.  

includeDeleted - If true the deleted file will be returned.  
\item {\bf Returns:}  

file the read file.  
\item {\bf Throws:} CmsException  

if the user has not the rights to read this resource, or if the file couldn't
be read.  
\end{description}

\end{description}

o {\bf readFile} 

\begin{PRE}
 public CmsFile readFile(String folder,
                         String filename) throws CmsException
\end{PRE}

\begin{description}
\htmlDD Reads a file from the Cms. 

\begin{description}
\item {\bf Parameters:}  

folder - the complete path to the folder from which the file will be read.  

filename - the name of the file to be read.  
\item {\bf Returns:}  

file the read file.  
\item {\bf Throws:} CmsException  

, if the user has not the rights to read this resource, or if the file
couldn't be read.  
\end{description}

\end{description}

o {\bf readFileExtensions} 

\begin{PRE}
 public Hashtable readFileExtensions() throws CmsException
\end{PRE}

\begin{description}
\htmlDD Gets the known file extensions (=suffixes). 

\begin{description}
\item {\bf Returns:}  

a Hashtable with all known file extensions as Strings.  
\item {\bf Throws:} CmsException  

if operation was not successful.  
\end{description}

\end{description}

o {\bf readFileHeader} 

\begin{PRE}
 public CmsResource readFileHeader(String filename) throws CmsException
\end{PRE}

\begin{description}
\htmlDD Reads a file header from the Cms. \htmlBR
The reading excludes the filecontent. 

\begin{description}
\item {\bf Parameters:}  

filename - the complete path of the file to be read.  
\item {\bf Returns:}  

file the read file.  
\item {\bf Throws:} CmsException  

, if the user has not the rights to read the file headers, or if the file
headers couldn't be read.  
\end{description}

\end{description}

o {\bf readFileHeader} 

\begin{PRE}
 public CmsResource readFileHeader(String filename,
                                   int projectId) throws CmsException
\end{PRE}

\begin{description}
\htmlDD Reads a file header from the Cms. \htmlBR
The reading excludes the filecontent. 

\begin{description}
\item {\bf Parameters:}  

filename - the complete path of the file to be read.  

projectId - the id of the project where the resource should belong to  
\item {\bf Returns:}  

file the read file.  
\item {\bf Throws:} CmsException  

, if the user has not the rights to read the file headers, or if the file
headers couldn't be read.  
\end{description}

\end{description}

o {\bf readFileHeader} 

\begin{PRE}
 public CmsResource readFileHeader(String folder,
                                   String filename) throws CmsException
\end{PRE}

\begin{description}
\htmlDD Reads a file header from the Cms. \htmlBR
The reading excludes the filecontent. 

\begin{description}
\item {\bf Parameters:}  

folder - the complete path to the folder from which the file will be read.  

filename - the name of the file to be read.  
\item {\bf Returns:}  

file the read file.  
\item {\bf Throws:} CmsException  

if the user has not the rights to read the file header, or if the file header
couldn't be read.  
\end{description}

\end{description}

o {\bf readFileHeaderForHist} 

\begin{PRE}
 public CmsResource readFileHeaderForHist(String filename,
                                          int versionId) throws CmsException
\end{PRE}

\begin{description}
\htmlDD Reads a file header from the Cms for history. \htmlBR
The reading excludes the filecontent. 

\begin{description}
\item {\bf Parameters:}  

filename - the complete path of the file to be read.  

versionId - the version id of the resource  
\item {\bf Returns:}  

file the read file.  
\item {\bf Throws:} CmsException  

, if the user has not the rights to read the file headers, or if the file
headers couldn't be read.  
\end{description}

\end{description}

o {\bf readFileForHist} 

\begin{PRE}
 public CmsBackupResource readFileForHist(String filename,
                                          int versionId) throws CmsException
\end{PRE}

\begin{description}
\htmlDD Reads a file from the Cms for history. \htmlBR
The reading includes the filecontent. 

\begin{description}
\item {\bf Parameters:}  

filename - the complete path of the file to be read.  

versionId - the version id of the resource  
\item {\bf Returns:}  

file the read file.  
\item {\bf Throws:} CmsException  

, if the user has not the rights to read the file, or if the file couldn't be
read.  
\end{description}

\end{description}

o {\bf readFileHeaders} 

\begin{PRE}
 public Vector readFileHeaders(int projectId) throws CmsException
\end{PRE}

\begin{description}
\htmlDD Reads all file headers of a project from the Cms. 

\begin{description}
\item {\bf Parameters:}  

projectId - the id of the project to read the file headers for.  
\item {\bf Returns:}  

a Vector of resources.  
\item {\bf Throws:} CmsException  

if the user has not the rights to read the file headers, or if the file
headers couldn't be read.  
\end{description}

\end{description}

o {\bf readFolder} 

\begin{PRE}
 public CmsFolder readFolder(String folder) throws CmsException
\end{PRE}

\begin{description}
\htmlDD Reads a folder from the Cms. 

\begin{description}
\item {\bf Parameters:}  

folder - the complete path to the folder to be read.  
\item {\bf Returns:}  

folder the read folder.  
\item {\bf Throws:} CmsException  

if the user has not the rights to read this resource, or if the folder
couldn't be read.  
\end{description}

\end{description}

o {\bf readFolder} 

\begin{PRE}
 public CmsFolder readFolder(String folder,
                             boolean includeDeleted) throws CmsException
\end{PRE}

\begin{description}
\htmlDD Reads a folder from the Cms. 

\begin{description}
\item {\bf Parameters:}  

folder - the complete path to the folder to be read.  

includeDeleted - Include the folder if it is marked as deleted  
\item {\bf Returns:}  

folder the read folder.  
\item {\bf Throws:} CmsException  

if the user has not the rights to read this resource, or if the folder
couldn't be read.  
\end{description}

\end{description}

o {\bf readFolder} 

\begin{PRE}
 public CmsFolder readFolder(String folder,
                             String folderName) throws CmsException
\end{PRE}

\begin{description}
\htmlDD Reads a folder from the Cms. 

\begin{description}
\item {\bf Parameters:}  

folder - the complete path to the folder from which the folder will be read.  

foldername - the name of the folder to be read.  
\item {\bf Returns:}  

folder the read folder.  
\item {\bf Throws:} CmsException  

if the user has not the rights to read this resource, or if the folder
couldn't be read.  
\end{description}

\end{description}

o {\bf readFolder} 

\begin{PRE}
 public CmsFolder readFolder(int folderid,
                             boolean includeDeleted) throws CmsException
\end{PRE}

\begin{description}
\htmlDD Reads a folder from the Cms. 

\begin{description}
\item {\bf Parameters:}  

folderid - the id of the folder to be read.  

includeDeleted - Include the folder if it is marked as deleted  
\item {\bf Returns:}  

folder the read folder.  
\item {\bf Throws:} CmsException  

if the user has not the rights to read this resource, or if the folder
couldn't be read.  
\end{description}

\end{description}

o {\bf readGivenTasks} 

\begin{PRE}
 public Vector readGivenTasks(int projectId,
                              String ownerName,
                              int taskType,
                              String orderBy,
                              String sort) throws CmsException
\end{PRE}

\begin{description}
\htmlDD Reads all given tasks from a user for a project. 

\begin{description}
\item {\bf Parameters:}  

projectId - the id of the project in which the tasks are defined.  

owner - the owner of the task.  

tasktype - the type of task you want to read: C\_TASKS\_ALL, C\_TASKS\_OPEN,
C\_TASKS\_DONE, C\_TASKS\_NEW.  

orderBy - specifies how to order the tasks.  
\item {\bf Throws:} CmsException  

if operation was not successful.  
\end{description}

\end{description}

o {\bf readGroup} 

\begin{PRE}
 public CmsGroup readGroup(CmsProject project) throws CmsException
\end{PRE}

\begin{description}
\htmlDD Reads the group of a project from the OpenCms. 

\begin{description}
\item {\bf Returns:}  

the group of the given project.  
\item {\bf Throws:} CmsException  

if operation was not successful.  
\end{description}

\end{description}

o {\bf readGroup} 

\begin{PRE}
 public CmsGroup readGroup(CmsResource resource) throws CmsException
\end{PRE}

\begin{description}
\htmlDD Reads the group of a resource from the Cms. 

\begin{description}
\item {\bf Returns:}  

the group of a resource.  
\item {\bf Throws:} CmsException  

if operation was not successful.  
\end{description}

\end{description}

o {\bf readGroup} 

\begin{PRE}
 public CmsGroup readGroup(CmsTask task) throws CmsException
\end{PRE}

\begin{description}
\htmlDD Reads the group (role) of a task from the Cms. 

\begin{description}
\item {\bf Parameters:}  

task - the task to read the role from.  
\item {\bf Returns:}  

the group of the task.  
\item {\bf Throws:} CmsException  

if operation was not successful.  
\end{description}

\end{description}

o {\bf readGroup} 

\begin{PRE}
 public CmsGroup readGroup(String groupname) throws CmsException
\end{PRE}

\begin{description}
\htmlDD Reads a group of the Cms. 

\begin{description}
\item {\bf Parameters:}  

groupname - the name of the group to be returned.  
\item {\bf Returns:}  

a group in the Cms.  
\item {\bf Throws:} CmsException  

if operation was not successful.  
\end{description}

\end{description}

o {\bf readGroup} 

\begin{PRE}
 public CmsGroup readGroup(int groupid) throws CmsException
\end{PRE}

\begin{description}
\htmlDD Reads a group of the Cms. 

\begin{description}
\item {\bf Parameters:}  

groupid - the id of the group to be returned.  
\item {\bf Returns:}  

a group in the Cms.  
\item {\bf Throws:} CmsException  

if operation was not successful.  
\end{description}

\end{description}

o {\bf readManagerGroup} 

\begin{PRE}
 public CmsGroup readManagerGroup(CmsProject project) throws CmsException
\end{PRE}

\begin{description}
\htmlDD Reads the managergroup of a project from the Cms. 

\begin{description}
\item {\bf Returns:}  

the managergroup of a project.  
\item {\bf Throws:} CmsException  

if operation was not successful.  
\end{description}

\end{description}

o {\bf readMimeTypes} 

\begin{PRE}
 public Hashtable readMimeTypes() throws CmsException
\end{PRE}

\begin{description}
\htmlDD Gets all Mime-Types known by the system. 

\begin{description}
\item {\bf Returns:}  

a Hashtable containing all mime-types.  
\item {\bf Throws:} CmsException  

if operation was not successful.  
\end{description}

\end{description}

o {\bf readOriginalAgent} 

\begin{PRE}
 public CmsUser readOriginalAgent(CmsTask task) throws CmsException
\end{PRE}

\begin{description}
\htmlDD Reads the original agent of a task from the Cms. 

\begin{description}
\item {\bf Parameters:}  

task - the task to read the original agent from.  
\item {\bf Returns:}  

the owner of a task.  
\item {\bf Throws:} CmsException  

if operation was not successful.  
\end{description}

\end{description}

o {\bf readOwner} 

\begin{PRE}
 public CmsUser readOwner(CmsProject project) throws CmsException
\end{PRE}

\begin{description}
\htmlDD Reads the owner of a project from the Cms. 

\begin{description}
\item {\bf Returns:}  

the owner of the given project.  
\item {\bf Throws:} CmsException  

if operation was not successful.  
\end{description}

\end{description}

o {\bf readOwner} 

\begin{PRE}
 public CmsUser readOwner(CmsResource resource) throws CmsException
\end{PRE}

\begin{description}
\htmlDD Reads the owner of a resource from the Cms. 

\begin{description}
\item {\bf Returns:}  

the owner of a resource.  
\item {\bf Throws:} CmsException  

if operation was not successful.  
\end{description}

\end{description}

o {\bf readOwner} 

\begin{PRE}
 public CmsUser readOwner(CmsTask task) throws CmsException
\end{PRE}

\begin{description}
\htmlDD Reads the owner (initiator) of a task from the Cms. 

\begin{description}
\item {\bf Parameters:}  

tasktThe - task to read the owner from.  
\item {\bf Returns:}  

the owner of a task.  
\item {\bf Throws:} CmsException  

if operation was not successful.  
\end{description}

\end{description}

o {\bf readOwner} 

\begin{PRE}
 public CmsUser readOwner(CmsTaskLog log) throws CmsException
\end{PRE}

\begin{description}
\htmlDD Reads the owner of a tasklog from the Cms. 

\begin{description}
\item {\bf Returns:}  

the owner of a resource.  
\item {\bf Throws:} CmsException  

if operation was not successful.  
\end{description}

\end{description}

o {\bf readProject} 

\begin{PRE}
 public CmsProject readProject(int id) throws CmsException
\end{PRE}

\begin{description}
\htmlDD Reads a project from the Cms. 

\begin{description}
\item {\bf Parameters:}  

task - the task for which the project will be read.  
\item {\bf Throws:} CmsException  

if operation was not successful.  
\end{description}

\end{description}

o {\bf readProject} 

\begin{PRE}
 public CmsProject readProject(CmsResource res) throws CmsException
\end{PRE}

\begin{description}
\htmlDD Reads a project from the Cms. 

\begin{description}
\item {\bf Parameters:}  

id - the id of the project to read.  
\item {\bf Throws:} CmsException  

if operation was not successful.  
\end{description}

\end{description}

o {\bf readProject} 

\begin{PRE}
 public CmsProject readProject(CmsTask task) throws CmsException
\end{PRE}

\begin{description}
\htmlDD Reads a project from the Cms. 

\begin{description}
\item {\bf Parameters:}  

name - the resource for which the project will be read.  
\item {\bf Throws:} CmsException  

if operation was not successful.  
\end{description}

\end{description}

o {\bf readProjectView} 

\begin{PRE}
 public Vector readProjectView(int projectId,
                               String filter) throws CmsException
\end{PRE}

\begin{description}
\htmlDD Reads all file headers of a project from the Cms. 

\begin{description}
\item {\bf Parameters:}  

projectId - the id of the project to read the file headers for.  

filter - The filter for the resources (all, new, changed, deleted, locked)  
\item {\bf Returns:}  

a Vector of resources.  
\end{description}

\end{description}

o {\bf readBackupProject} 

\begin{PRE}
 public CmsBackupProject readBackupProject(int versionId) throws CmsException
\end{PRE}

\begin{description}
\htmlDD Reads a project from the Cms. 

\begin{description}
\item {\bf Parameters:}  

task - the task for which the project will be read.  
\item {\bf Throws:} CmsException  

if operation was not successful.  
\end{description}

\end{description}

o {\bf readProjectLogs} 

\begin{PRE}
 public Vector readProjectLogs(int projectId) throws CmsException
\end{PRE}

\begin{description}
\htmlDD Reads log entries for a project. 

\begin{description}
\item {\bf Parameters:}  

projectId - the id of the project for which the tasklog will be read.  
\item {\bf Returns:}  

a Vector of new TaskLog objects  
\item {\bf Throws:} CmsException  

if operation was not successful.  
\end{description}

\end{description}

o {\bf readProperty} 

\begin{PRE}
 public String readProperty(String name,
                            String property) throws CmsException
\end{PRE}

\begin{description}
\htmlDD Returns a Property of a file or folder. 

\begin{description}
\item {\bf Parameters:}  

name - the resource-name for which the property will be read.  

property - the property-definition name of the property that will be read.  
\item {\bf Returns:}  

property the Property as string.  
\item {\bf Throws:} CmsException  

if operation was not successful  
\end{description}

\end{description}

o {\bf readPropertydefinition} 

\begin{PRE}
 public CmsPropertydefinition readPropertydefinition(String name,
                                                     String resourcetype) throws CmsException
\end{PRE}

\begin{description}
\htmlDD Reads the property-definition for the resource type. 

\begin{description}
\item {\bf Parameters:}  

name - the name of the property-definition to read.  

resourcetype - the name of the resource type for the property-definition.  
\item {\bf Returns:}  

the property-definition.  
\item {\bf Throws:} CmsException  

if operation was not successful.  
\end{description}

\end{description}

o {\bf readTask} 

\begin{PRE}
 public CmsTask readTask(int id) throws CmsException
\end{PRE}

\begin{description}
\htmlDD Reads the task with the given id. 

\begin{description}
\item {\bf Parameters:}  

id - the id of the task to be read.  
\item {\bf Throws:} CmsException  

if operation was not successful.  
\end{description}

\end{description}

o {\bf readTaskLogs} 

\begin{PRE}
 public Vector readTaskLogs(int taskid) throws CmsException
\end{PRE}

\begin{description}
\htmlDD Reads log entries for a task. 

\begin{description}
\item {\bf Parameters:}  

taskid - the task for which the tasklog will be read.  
\item {\bf Returns:}  

a Vector of new TaskLog objects.  
\item {\bf Throws:} CmsException  

if operation was not successful.  
\end{description}

\end{description}

o {\bf readTasksForProject} 

\begin{PRE}
 public Vector readTasksForProject(int projectId,
                                   int tasktype,
                                   String orderBy,
                                   String sort) throws CmsException
\end{PRE}

\begin{description}
\htmlDD Reads all tasks for a project. 

\begin{description}
\item {\bf Parameters:}  

projectId - the id of the project in which the tasks are defined. Can be null
to select all tasks.  

orderBy - specifies how to order the tasks.  

sort - sort order: C\_SORT\_ASC, C\_SORT\_DESC, or null.  
\item {\bf Throws:} CmsException  

if operation was not successful.  
\end{description}

\end{description}

o {\bf readTasksForRole} 

\begin{PRE}
 public Vector readTasksForRole(int projectId,
                                String roleName,
                                int tasktype,
                                String orderBy,
                                String sort) throws CmsException
\end{PRE}

\begin{description}
\htmlDD Reads all tasks for a role in a project. 

\begin{description}
\item {\bf Parameters:}  

projectId - the id of the Project in which the tasks are defined.  

user - the user who has to process the task.  

tasktype - the type of task you want to read: C\_TASKS\_ALL, C\_TASKS\_OPEN,
C\_TASKS\_DONE, C\_TASKS\_NEW.  

orderBy - specifies how to order the tasks.  

sort - sort order C\_SORT\_ASC, C\_SORT\_DESC, or null  
\item {\bf Throws:} CmsException  

if operation was not successful.  
\end{description}

\end{description}

o {\bf readTasksForUser} 

\begin{PRE}
 public Vector readTasksForUser(int projectId,
                                String userName,
                                int tasktype,
                                String orderBy,
                                String sort) throws CmsException
\end{PRE}

\begin{description}
\htmlDD Reads all tasks for a user in a project. 

\begin{description}
\item {\bf Parameters:}  

projectId - the id of the Project in which the tasks are defined.  

role - the user who has to process the task.  

tasktype - the type of task you want to read: C\_TASKS\_ALL, C\_TASKS\_OPEN,
C\_TASKS\_DONE, C\_TASKS\_NEW.  

orderBy - specifies how to order the tasks.  

sort - sort order C\_SORT\_ASC, C\_SORT\_DESC, or null  
\item {\bf Throws:} CmsException  

if operation was not successful.  
\end{description}

\end{description}

o {\bf readUser} 

\begin{PRE}
 public CmsUser readUser(int id) throws CmsException
\end{PRE}

\begin{description}
\htmlDD Returns a user in the Cms. 

\begin{description}
\item {\bf Parameters:}  

id - the id of the user to be returned.  
\item {\bf Returns:}  

a user in the Cms.  
\item {\bf Throws:} CmsException  

if operation was not successful  
\end{description}

\end{description}

o {\bf readUser} 

\begin{PRE}
 public CmsUser readUser(String username) throws CmsException
\end{PRE}

\begin{description}
\htmlDD Returns a user in the Cms. 

\begin{description}
\item {\bf Parameters:}  

username - the name of the user to be returned.  
\item {\bf Returns:}  

a user in the Cms.  
\item {\bf Throws:} CmsException  

if operation was not successful  
\end{description}

\end{description}

o {\bf readUser} 

\begin{PRE}
 public CmsUser readUser(String username,
                         int type) throws CmsException
\end{PRE}

\begin{description}
\htmlDD Returns a user in the Cms. 

\begin{description}
\item {\bf Parameters:}  

username - the name of the user to be returned.  

type - the type of the user.  
\item {\bf Returns:}  

a user in the Cms.  
\item {\bf Throws:} CmsException  

if operation was not successful  
\end{description}

\end{description}

o {\bf readUser} 

\begin{PRE}
 public CmsUser readUser(String username,
                         String password) throws CmsException
\end{PRE}

\begin{description}
\htmlDD Returns a user in the Cms, if the password is correct. 

\begin{description}
\item {\bf Parameters:}  

username - the name of the user to be returned.  

password - the password of the user to be returned.  
\item {\bf Returns:}  

a user in the Cms.  
\item {\bf Throws:} CmsException  

if operation was not successful  
\end{description}

\end{description}

o {\bf readWebUser} 

\begin{PRE}
 public CmsUser readWebUser(String username) throws CmsException
\end{PRE}

\begin{description}
\htmlDD Returns a user object if the password for the user is correct. {\bf
Security:} All users are granted. 

\begin{description}
\item {\bf Parameters:}  

currentUser - The user who requested this method.  

currentProject - The current project of the user.  

username - The username of the user that is to be read.  
\item {\bf Returns:}  

User  
\item {\bf Throws:} CmsException  

Throws CmsException if operation was not succesful  
\end{description}

\end{description}

o {\bf readWebUser} 

\begin{PRE}
 public CmsUser readWebUser(String username,
                            String password) throws CmsException
\end{PRE}

\begin{description}
\htmlDD Returns a user object if the password for the user is correct. {\bf
Security:} All users are granted. 

\begin{description}
\item {\bf Parameters:}  

currentUser - The user who requested this method.  

currentProject - The current project of the user.  

username - The username of the user that is to be read.  

password - The password of the user that is to be read.  
\item {\bf Returns:}  

User  
\item {\bf Throws:} CmsException  

Throws CmsException if operation was not succesful  
\end{description}

\end{description}

o {\bf reaktivateTask} 

\begin{PRE}
 public void reaktivateTask(int taskId) throws CmsException
\end{PRE}

\begin{description}
\htmlDD Reactivates a task from the Cms. 

\begin{description}
\item {\bf Parameters:}  

taskid - the Id of the task to accept.  
\item {\bf Throws:} CmsException  

if operation was not successful.  
\end{description}

\end{description}

o {\bf recoverPassword} 

\begin{PRE}
 public void recoverPassword(String username,
                             String recoveryPassword,
                             String newPassword) throws CmsException
\end{PRE}

\begin{description}
\htmlDD Sets a new password if the user knows his recovery-password. 

\begin{description}
\item {\bf Parameters:}  

username - the name of the user.  

recoveryPassword - the recovery password.  

newPassword - the new password.  
\item {\bf Throws:} CmsException  

if operation was not successfull.  
\end{description}

\end{description}

o {\bf removeUserFromGroup} 

\begin{PRE}
 public void removeUserFromGroup(String username,
                                 String groupname) throws CmsException
\end{PRE}

\begin{description}
\htmlDD Removes a user from a group. 

{\bf Security:} Only the admin user is allowed to remove a user from a group. 

\begin{description}
\item {\bf Parameters:}  

username - the name of the user that is to be removed from the group.  

groupname - the name of the group.  
\item {\bf Throws:} CmsException  

if operation was not successful.  
\end{description}

\end{description}

o {\bf renameFile} 

\begin{PRE}
 public void renameFile(String oldname,
                        String newname) throws CmsException
\end{PRE}

\begin{description}
\htmlDD {\bf Note: renameFile() is deprecated.} {\it Use renameResource
instead.} 

Renames the file to the new name. 

\begin{description}
\item {\bf Parameters:}  

oldname - the complete path to the file which will be renamed.  

newname - the new name of the file.  
\item {\bf Throws:} CmsException  

if the user has not the rights to rename the file, or if the file couldn't be
renamed.  
\end{description}

\end{description}

o {\bf renameResource} 

\begin{PRE}
 public void renameResource(String oldname,
                            String newname) throws CmsException
\end{PRE}

\begin{description}
\htmlDD Renames the resource to the new name. 

\begin{description}
\item {\bf Parameters:}  

oldname - the complete path to the file which will be renamed.  

newname - the new name of the file.  
\item {\bf Throws:} CmsException  

if the user has not the rights to rename the file, or if the file couldn't be
renamed.  
\end{description}

\end{description}

o {\bf doRenameFile} 

\begin{PRE}
 protected void doRenameFile(String oldname,
                             String newname) throws CmsException
\end{PRE}

\begin{description}
\htmlDD Renames the resource to the new name. 

\begin{description}
\item {\bf Parameters:}  

oldname - the complete path to the file which will be renamed.  

newname - the new name of the file.  
\item {\bf Throws:} CmsException  

if the user has not the rights to rename the file, or if the file couldn't be
renamed.  
\end{description}

\end{description}

o {\bf restoreResource} 

\begin{PRE}
 public void restoreResource(int versionId,
                             String filename) throws CmsException
\end{PRE}

\begin{description}
\htmlDD Restores a file in the current project with a version in the backup 

\begin{description}
\item {\bf Parameters:}  

versionId - The version id of the resource  

filename - The name of the file to restore  
\item {\bf Throws:} CmsException  

Throws CmsException if operation was not succesful.  
\end{description}

\end{description}

o {\bf doRestoreResource} 

\begin{PRE}
 protected void doRestoreResource(int versionId,
                                  String filename) throws CmsException
\end{PRE}

\begin{description}
\htmlDD Restores a file in the current project with a version in the backup 

\begin{description}
\item {\bf Parameters:}  

versionId - The version id of the resource  

filename - The name of the file to restore  
\item {\bf Throws:} CmsException  

Throws CmsException if operation was not succesful.  
\end{description}

\end{description}

o {\bf rootFolder} 

\begin{PRE}
 public CmsFolder rootFolder() throws CmsException
\end{PRE}

\begin{description}
\htmlDD Returns the root-folder object. 

\begin{description}
\item {\bf Returns:}  

the root-folder object.  
\item {\bf Throws:} CmsException  

if operation was not successful.  
\end{description}

\end{description}

o {\bf setLauncherManager} 

\begin{PRE}
 public void setLauncherManager(CmsLauncherManager newM\_launcherManager)
\end{PRE}

\begin{description}
\htmlDD Set the launcher manager used with this instance of CmsObject.
Creation date: (10/23/00 14:50:15) 

\begin{description}
\item {\bf Parameters:}  

newM\_launcherManager - com.opencms.launcher.CmsLauncherManager  
\end{description}

\end{description}

o {\bf setName} 

\begin{PRE}
 public void setName(int taskId,
                     String name) throws CmsException
\end{PRE}

\begin{description}
\htmlDD Set a new name for a task. 

\begin{description}
\item {\bf Parameters:}  

taskid - the id of the task.  

name - the new name of the task.  
\item {\bf Throws:} CmsException  

if operationwas not successful.  
\end{description}

\end{description}

o {\bf setParentGroup} 

\begin{PRE}
 public void setParentGroup(String groupName,
                            String parentGroupName) throws CmsException
\end{PRE}

\begin{description}
\htmlDD Sets a new parent-group for an already existing group in the Cms. 

\begin{description}
\item {\bf Parameters:}  

groupName - the name of the group that should be written to the Cms.  

parentGroupName - the name of the parentGroup to set, or null if the parent
group should be deleted.  
\item {\bf Throws:} CmsException  

if operation was not successfull.  
\end{description}

\end{description}

o {\bf setPassword} 

\begin{PRE}
 public void setPassword(String username,
                         String newPassword) throws CmsException
\end{PRE}

\begin{description}
\htmlDD Sets the password for a user. 

\begin{description}
\item {\bf Parameters:}  

username - the name of the user.  

newPassword - the new password.  
\item {\bf Throws:} CmsException  

if operation was not successful.  
\end{description}

\end{description}

o {\bf setPassword} 

\begin{PRE}
 public void setPassword(String username,
                         String oldPassword,
                         String newPassword) throws CmsException
\end{PRE}

\begin{description}
\htmlDD Sets the password for a user. 

\begin{description}
\item {\bf Parameters:}  

username - the name of the user.  

oldPassword - the old password.  

newPassword - the new password.  
\item {\bf Throws:} CmsException  

if operation was not successful.  
\end{description}

\end{description}

o {\bf setPriority} 

\begin{PRE}
 public void setPriority(int taskId,
                         int priority) throws CmsException
\end{PRE}

\begin{description}
\htmlDD Sets the priority of a task. 

\begin{description}
\item {\bf Parameters:}  

taskid - the id of the task.  

priority - the new priority value.  
\item {\bf Throws:} CmsException  

if operation was not successful.  
\end{description}

\end{description}

o {\bf setRecoveryPassword} 

\begin{PRE}
 public void setRecoveryPassword(String username,
                                 String oldPassword,
                                 String newPassword) throws CmsException
\end{PRE}

\begin{description}
\htmlDD Sets the recovery password for a user. 

\begin{description}
\item {\bf Parameters:}  

username - the name of the user.  

password - the password.  

newPassword - the new recovery password.  
\item {\bf Throws:} CmsException  

if operation was not successful.  
\end{description}

\end{description}

o {\bf setTaskPar} 

\begin{PRE}
 public void setTaskPar(int taskid,
                        String parname,
                        String parvalue) throws CmsException
\end{PRE}

\begin{description}
\htmlDD Set a parameter for a task. 

\begin{description}
\item {\bf Parameters:}  

taskid - the Id of the task.  

parname - the ame of the parameter.  

parvalue - the value of the parameter.  
\item {\bf Returns:}  

the id of the inserted parameter or 0 if the parameter already exists for this
task.  
\item {\bf Throws:} CmsException  

if operation was not successful.  
\end{description}

\end{description}

o {\bf setTimeout} 

\begin{PRE}
 public void setTimeout(int taskId,
                        long timeout) throws CmsException
\end{PRE}

\begin{description}
\htmlDD Sets the timeout of a task. 

\begin{description}
\item {\bf Parameters:}  

taskid - the id of the task.  

timeout - the new timeout value.  
\item {\bf Throws:} CmsException  

if operation was not successful.  
\end{description}

\end{description}

o {\bf syncFolder} 

\begin{PRE}
 public void syncFolder(String resourceName) throws CmsException
\end{PRE}

\begin{description}
\htmlDD Synchronize cms-resources on virtual filesystem with the server
filesystem. 

\begin{description}
\item {\bf Parameters:}  

syncFile - the name (absolute Path) of the resource that should be
synchronized.  

syncPath - the name of path on server filesystem where the resource should be
synchronized.  
\item {\bf Throws:} CmsException  

if operation was not successful.  
\end{description}

\end{description}

o {\bf unlockProject} 

\begin{PRE}
 public void unlockProject(int id) throws CmsException
\end{PRE}

\begin{description}
\htmlDD Unlocks all resources of a project. 

\begin{description}
\item {\bf Parameters:}  

id - the id of the project to be unlocked.  
\item {\bf Throws:} CmsException  

if operation was not successful.  
\end{description}

\end{description}

o {\bf unlockResource} 

\begin{PRE}
 public void unlockResource(String resource) throws CmsException
\end{PRE}

\begin{description}
\htmlDD Unlocks a resource. \htmlBR
A user can unlock a resource, so other users may lock this file. 

\begin{description}
\item {\bf Parameters:}  

resource - the complete path to the resource to be unlocked.  
\item {\bf Throws:} CmsException  

if the user has not the rights to unlock this resource.  
\end{description}

\end{description}

o {\bf undoChanges} 

\begin{PRE}
 public void undoChanges(String filename) throws CmsException
\end{PRE}

\begin{description}
\htmlDD Undo changes in a file by copying the online file. 

\begin{description}
\item {\bf Parameters:}  

filename - the complete path of the file.  
\item {\bf Throws:} CmsException  

if the file couldn't be deleted, or if the user has not the appropriate rights
to write the file.  
\end{description}

\end{description}

o {\bf doUndoChanges} 

\begin{PRE}
 protected void doUndoChanges(String resource) throws CmsException
\end{PRE}

\begin{description}
\htmlDD Undo changes in a file. \htmlBR

\begin{description}
\item {\bf Parameters:}  

resource - the complete path to the resource to be unlocked.  
\item {\bf Throws:} CmsException  

if the user has not the rights to write this resource.  
\end{description}

\end{description}

o {\bf doUnlockResource} 

\begin{PRE}
 protected void doUnlockResource(String resource) throws CmsException
\end{PRE}

\begin{description}
\htmlDD Unlocks a resource. \htmlBR
A user can unlock a resource, so other users may lock this file. 

\begin{description}
\item {\bf Parameters:}  

resource - the complete path to the resource to be unlocked.  
\item {\bf Throws:} CmsException  

if the user has not the rights to unlock this resource.  
\end{description}

\end{description}

o {\bf userInGroup} 

\begin{PRE}
 public boolean userInGroup(String username,
                            String groupname) throws CmsException
\end{PRE}

\begin{description}
\htmlDD Tests, if a user is member of the given group. 

\begin{description}
\item {\bf Parameters:}  

username - the name of the user to test.  

groupname - the name of the group to test.  
\item {\bf Returns:}  

{\tt true}, if the user is in the group; {\tt else} false otherwise.  
\item {\bf Throws:} CmsException  

if operation was not successful.  
\end{description}

\end{description}

o {\bf version} 

\begin{PRE}
 public String version()
\end{PRE}

\begin{description}
\htmlDD Returns a String containing version information for this OpenCms. 

\begin{description}
\item {\bf Returns:}  

version a String containnig the version of OpenCms.  
\end{description}

\end{description}

o {\bf writeExportPath} 

\begin{PRE}
 public void writeExportPath(String path) throws CmsException
\end{PRE}

\begin{description}
\htmlDD Writes the export-path for the system. \htmlBR
This path is used for db-export and db-import. 

\begin{description}
\item {\bf Parameters:}  

mountpoint - the mount point in the Cms filesystem.  
\item {\bf Throws:} CmsException  

if operation ws not successful.  
\end{description}

\end{description}

o {\bf writeFile} 

\begin{PRE}
 public void writeFile(CmsFile file) throws CmsException
\end{PRE}

\begin{description}
\htmlDD Writes a file to the Cms. 

\begin{description}
\item {\bf Parameters:}  

file - the file to write.  
\item {\bf Throws:} CmsException  

if resourcetype is set to folder. The CmsException will also be thrown, if the
user has not the rights write the file.  
\end{description}

\end{description}

o {\bf writeFileExtensions} 

\begin{PRE}
 public void writeFileExtensions(Hashtable extensions) throws CmsException
\end{PRE}

\begin{description}
\htmlDD Writes the file extensions. 

{\bf Security:} Only the admin user is allowed to write file extensions. 

\begin{description}
\item {\bf Parameters:}  

extensions - holds extensions as keys and resourcetypes (Strings) as values.  
\end{description}

\end{description}

o {\bf writeFileHeader} 

\begin{PRE}
 public void writeFileHeader(CmsFile file) throws CmsException
\end{PRE}

\begin{description}
\htmlDD Writes a file-header to the Cms. 

\begin{description}
\item {\bf Parameters:}  

file - the file to write.  
\item {\bf Throws:} CmsException  

if resourcetype is set to folder. The CmsException will also be thrown, if the
user has not the rights to write the file header..  
\end{description}

\end{description}

o {\bf writeGroup} 

\begin{PRE}
 public void writeGroup(CmsGroup group) throws CmsException
\end{PRE}

\begin{description}
\htmlDD Writes an already existing group to the Cms. 

\begin{description}
\item {\bf Parameters:}  

group - the group that should be written to the Cms.  
\item {\bf Throws:} CmsException  

if operation was not successful.  
\end{description}

\end{description}

o {\bf writeProperties} 

\begin{PRE}
 public void writeProperties(String name,
                             Hashtable properties) throws CmsException
\end{PRE}

\begin{description}
\htmlDD Writes a couple of Properties for a file or folder. 

\begin{description}
\item {\bf Parameters:}  

name - the resource-name of which the Property has to be set.  

properties - a Hashtable with property-definitions and property values as
Strings.  
\item {\bf Throws:} CmsException  

if operation was not successful.  
\end{description}

\end{description}

o {\bf writeProperty} 

\begin{PRE}
 public void writeProperty(String name,
                           String property,
                           String value) throws CmsException
\end{PRE}

\begin{description}
\htmlDD Writes a property for a file or folder. 

\begin{description}
\item {\bf Parameters:}  

name - the resource-name for which the property will be set.  

property - the property-definition name.  

value - the value for the property to be set.  
\item {\bf Throws:} CmsException  

if operation was not successful.  
\end{description}

\end{description}

o {\bf writePropertydefinition} 

\begin{PRE}
 public CmsPropertydefinition writePropertydefinition(CmsPropertydefinition definition) throws CmsException
\end{PRE}

\begin{description}
\htmlDD {\bf Note: writePropertydefinition() is deprecated.} {\it Do not use
this method any longer because there is no type of propertydefinition} 

Writes the property-definition for the resource type. 

\begin{description}
\item {\bf Parameters:}  

propertydef - the property-definition to be written.  
\item {\bf Throws:} CmsException  

if operation was not successful.  
\end{description}

\end{description}

o {\bf writeTaskLog} 

\begin{PRE}
 public void writeTaskLog(int taskid,
                          String comment) throws CmsException
\end{PRE}

\begin{description}
\htmlDD Writes a new user tasklog for a task. 

\begin{description}
\item {\bf Parameters:}  

taskid - the Id of the task.  

comment - the description for the log.  
\item {\bf Throws:} CmsException  

if operation was not successful.  
\end{description}

\end{description}

o {\bf writeTaskLog} 

\begin{PRE}
 public void writeTaskLog(int taskid,
                          String comment,
                          int taskType) throws CmsException
\end{PRE}

\begin{description}
\htmlDD Writes a new user tasklog for a task. 

\begin{description}
\item {\bf Parameters:}  

taskid - the Id of the task .  

comment - the description for the log  

tasktype - the type of the tasklog. User tasktypes must be greater than 100.  
\item {\bf Throws:} CmsException  

if operation was not successful.  
\end{description}

\end{description}

o {\bf writeUser} 

\begin{PRE}
 public void writeUser(CmsUser user) throws CmsException
\end{PRE}

\begin{description}
\htmlDD Updates the user information. 

{\bf Security:} Only the admin user is allowed to update the user information.


\begin{description}
\item {\bf Parameters:}  

user - the user to be written.  
\item {\bf Throws:} CmsException  

if operation was not successful.  
\end{description}

\end{description}

o {\bf writeWebUser} 

\begin{PRE}
 public void writeWebUser(CmsUser user) throws CmsException
\end{PRE}

\begin{description}
\htmlDD Updates the user information of a web user. \htmlBR
Only a web user can be updated this way. 

\begin{description}
\item {\bf Parameters:}  

user - the user to be written.  
\item {\bf Throws:} CmsException  

if operation was not successful.  
\end{description}

\end{description}

o {\bf sendBroadcastMessage} 

\begin{PRE}
 public void sendBroadcastMessage(String message) throws CmsException
\end{PRE}

\begin{description}
\htmlDD Returns a list of all currently logged in users. This method is only
allowed for administrators. 

\end{description}

o {\bf getLoggedInUsers} 

\begin{PRE}
 public Vector getLoggedInUsers() throws CmsException
\end{PRE}

\begin{description}
\htmlDD Returns a list of all currently logged in users. This method is only
allowed for administrators. 

\end{description}

o {\bf changeLockedInProject} 

\begin{PRE}
 protected void changeLockedInProject(int projectId,
                                      String resourcename) throws CmsException
\end{PRE}

\begin{description}
\htmlDD Changes the project-id of a resource to the new project for publishing
the resource directly 

\begin{description}
\item {\bf Parameters:}  

newProjectId - The new project-id  

resourcename - The name of the resource to change  
\end{description}

\end{description}

o {\bf doChangeLockedInProject} 

\begin{PRE}
 protected void doChangeLockedInProject(int projectId,
                                        String resourcename) throws CmsException
\end{PRE}

\begin{description}
\htmlDD Changes the project-id of a resource to the new project for publishing
the resource directly 

\begin{description}
\item {\bf Parameters:}  

newProjectId - The new project-id  

resourcename - The name of the resource to change  
\end{description}

\end{description}

o {\bf getSiteRoot} 

\begin{PRE}
 public String getSiteRoot(String resourcename)
\end{PRE}

\begin{description}
\htmlDD Returns the name of the current site root, e.g. /default/vfs 

\begin{description}
\item {\bf Parameters:}  

resourcename - The name of the resource  
\item {\bf Returns:}  

String The resourcename including its site root  
\end{description}

\end{description}

o {\bf getSiteName} 

\begin{PRE}
 public String getSiteName()
\end{PRE}

\begin{description}
\htmlDD Returns the name of the current site, e.g. /default 

\begin{description}
\item {\bf Returns:}  

String The site name  
\end{description}

\end{description}

o {\bf setContextToVfs} 

\begin{PRE}
 public void setContextToVfs()
\end{PRE}

\begin{description}
\htmlDD Sets the name of the current site root of the virtual file system 

\end{description}

o {\bf setContextToCos} 

\begin{PRE}
 public void setContextToCos()
\end{PRE}

\begin{description}
\htmlDD Sets the name of the current site root of the content objects system 

\end{description}

o {\bf setContextTo} 

\begin{PRE}
 public void setContextTo(String name)
\end{PRE}

\begin{description}
\htmlDD Sets the name of the current site root 

\begin{description}
\item {\bf Parameters:}  

name - The name of the context  
\end{description}

\end{description}

o {\bf isHistoryEnabled} 

\begin{PRE}
 public boolean isHistoryEnabled()
\end{PRE}

\begin{description}
\htmlDD Check if the history is enabled 

\begin{description}
\item {\bf Returns:}  

boolean Is true if history is enabled  
\end{description}

\end{description}

o {\bf getBackupVersionId} 

\begin{PRE}
 public int getBackupVersionId()
\end{PRE}

\begin{description}
\htmlDD Get the next version id for the published backup resources 

\begin{description}
\item {\bf Returns:}  

int The new version id  
\end{description}

\end{description}

o {\bf backupProject} 

\begin{PRE}
 public void backupProject(int projectId,
                           int versionId,
                           long publishDate) throws CmsException
\end{PRE}

\begin{description}
\htmlDD Creates a backup of the published project 

\begin{description}
\item {\bf Parameters:}  

project - The project in which the resource was published.  

projectresources - The resources of the project  

versionId - The version of the backup  

publishDate - The date of publishing  

userId - The id of the user who had published the project  
\item {\bf Throws:} CmsException  

Throws CmsException if operation was not succesful.  
\end{description}

\end{description}

o {\bf readCronTable} 

\begin{PRE}
 public String readCronTable() throws CmsException
\end{PRE}

\begin{description}
\htmlDD Gets the Crontable. {\bf Security:} All users are garnted 

\begin{description}
\item {\bf Returns:}  

the crontable.  
\end{description}

\end{description}

o {\bf writeCronTable} 

\begin{PRE}
 public void writeCronTable(String crontable) throws CmsException
\end{PRE}

\begin{description}
\htmlDD Writes the Crontable. {\bf Security:} Only a administrator can do this


\begin{description}
\item {\bf Parameters:}  

currentUser - The user who requested this method.  

currentProject - The current project of the user.  
\item {\bf Returns:}  

the crontable.  
\end{description}

\end{description}

o {\bf digest} 

\begin{PRE}
 public String digest(String value)
\end{PRE}

\begin{description}
\htmlDD Method to encrypt the passwords. 

\begin{description}
\item {\bf Parameters:}  

value - The value to encrypt.  
\item {\bf Returns:}  

The encrypted value.  
\end{description}

\end{description}

\htmlHR

\begin{PRE}
All Packages  Class Hierarchy  This Package  Previous  Next  Index
\end{PRE}

