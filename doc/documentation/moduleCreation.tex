\chapter{Importing an existing module}
%============================================================================
\section{How to import an existing module}

\begin{itemize}

\item Select Administration in the View selection of the workplace.
\begin{figure}[hbt]
\begin{center}
\includegraphics[width=\sgw]{pics/modules/import0}
\end{center}
\end{figure}

\item Select the {\em Online} project in the {\em Project} selection:
\begin{figure}[hbt]
\begin{center}
\includegraphics[width=\sgw]{pics/modules/import1}
\end{center}
\end{figure}

\item Open the {\em Module management}.

\item Click the {\em Upload module from} icon:
\begin{figure}[hbt]
\begin{center}
\includegraphics[width=\sgw]{pics/modules/import2}
\end{center}
\end{figure}

\item The module import wizard starts. On the first page of the wizard you can choose from where you want 
to upload the module's zip file:
\begin{figure}[hbt]
\begin{center}
\includegraphics[width=\sgw]{pics/modules/import3}
\end{center}
\end{figure}

\begin{itemize}

\item {\em Local computer}: use this option to upload the module anywhere from your local filesystem 
(using http upload). On the next wizard page a file search dialog is shown and you can select the 
module's zip file in the local file system of your computer. 

\item {\em Server}: use this option to upload the module from the {\tt WEB-INF/export/modules/} directory of 
your OpenCms web application on the server.
On the next wizard page you get a selection with the package names of all modules found in 
{\tt WEB-INF/export/modules/} to choose from. 

\end{itemize}

\item Finally, click {\em Continue}. The import starts, reporting which files and folders are imported.

\item You might get an error or exception message during the import that tells you that you should do a 
restart afterwards. If so, do what you are told, and restart the OpenCms server after the last step. Be 
aware that you might have to restart the OpenCms server even if no such message is shown, see below for 
more details.

\item After you have left the import dialog the wizard ends and you should find the imported module in 
the list of all modules. 

\end{itemize}

{\bf Please note:} Some modules contain Java archives (JARs), class files or other resources that are automatically 
copied to the WEB-INF/classes/ or WEB-INF/lib/ folder of the OpenCms web application during the module import. 
Such modules sometimes require a restart of the OpenCms Servlet container so that the Java Classloader can 
load these new classes or resources.

The individual module documentation should contain a note if a module requires a server restart after 
installation. You will not always get an exception message during import if the module requires a server 
restart. In case you upload several modules, one server restart is usually enough after uploading all modules.