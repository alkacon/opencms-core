\chapter{Installation of OpenCms work release 4.3.x}
%-ltoh-   title := "OpenCms 4.3 - Installation"
During development the installation procedure coninuously
changes. If you downloaded the most recent work release
instead of the major release some steps for setting up a running
system differ from the description in chapter \ref{42install}.

\section{Installing the Apache Web Server}

Installation steps didn't change since the OpenCms major release 4.2. Please
see section \ref{42apache} for more details.

\section{Installing Java JDK and JSDK}

Installation steps didn't change since the OpenCms major release 4.2. Please
see section \ref{42jdk} for more details.

\section{Installing MySQL}

Installation steps didn't change since the OpenCms major release 4.2. Please
see section \ref{42mysql} for more details.

\section{Installing the OpenCms files}
We currently do not have an RPM or setup.exe, 
but there are plans to make them available in the near future.
We recommend that you unpack your OpenCms distribution archive file\\ 
(opencms\_4.3.yy from 
\rqhttp{http://www.opencms.com/servlets/opencms/service/download.html}{http://www.opencms.com/servlets/opencms/service/download.html})
in your \texttt{C:$\backslash$\backbug{Program Files}$\backslash$} folder on Windows systems and 
in the \texttt{/opt} folder on UNIX systems. 

While extracting the ZIP archive
a new folder \texttt{opencms\_4.3.yy} (with \texttt{yy}
replaced by the current build number) will be created. This new folder will be your OpenCms home
directory. Make it to your current directory and check the content of
the subfolder \texttt{oclib}. This folder is the right place for all external
components and drivers used by OpenCms. In order to run OpenCms properly
you should put the following JAR files in this folder:

\begin{itemize}
\item \texttt{xerces.jar} from the Apache Xerces XML parser. 
\item The MySQL JDBC database driver \texttt{mm.mysql-2.0.1-bin.jar}.
\item \texttt{mail.jar} and \texttt{activation.jar} from the JavaMail distribution.
\item \texttt{fop\_bin.jar} (optional) from the FOP packet of the Apache XML project.
\end{itemize}

See the Components page on the OpenCms web site\\
(\rqhttp{http://www.opencms.com/servlets/opencms/background/components.html}
{http://www.opencms.com/servlets/opencms/background/components.html}) 
or chapter \ref{components} of the OpenCms documentation for a more detailed description
of these external parts of OpenCms.


\section{Setting up the OpenCms database}
Installation stepsx didn't change since the OpenCms major release 4.2. Please
see section \ref{42dbsetup} for more details.

\section{Initializing the OpenCms database}
Edit the \texttt{opencms.properties} file in the config folder of your OpenCms home 
directory and set the path for the log file:

\begin{quote}
\texttt{log.file=(your OpenCms home directory)/logs/opencms.log}
\end{quote}

On UNIX systems, Apache may have problems writing to the logs folder. 
For this reason you should set its owner to nobody. 

OpenCms is able to export folders to the real file system on the server during 
publishing operations. This feature commonly is used for storing pictures or large downloadable
binary files in the Apache Web Server's
document root. This will increase performance while displaying web pages with OpenCms.
We recommend to set the following export paths in the \texttt{opencms.properties} file:

\begin{quote}
\begin{verbatim}
exportpoint.0 = /pics/
exportpoint.path.0 = (real path to apache document root)/pics/
exportpoint.1 = /download/
exportpoint.path.1 = (real path to apache document root)/download/
exportpoint.2 = /system/workplace/pics/
exportpoint.path.2 = (real path to apache document root)/pics/system/
\end{verbatim}
\end{quote}

Note the trailing slash
characters. They are very important and must not be omitted.

Ensure that all paths you are using as export points in your real file system
do really exist. OpenCms won't try to create these folders.

Similar to the log file on UNIX systems access rights conflicts may occur while
OpenCms is trying to write to these folders. Set their owner to nobody, too.

Now take a look at the file \texttt{cmssetup.txt}:
Set the path for OpenCms export files appropriate to your system by editing
the line beginning with "\texttt{writeExportPath}" to 

\begin{quote}
\texttt{writeExportPath "(your OpenCms home directory)/export/"}
\end{quote}

(you will find these lines near to the end of the file).
Again the trailing slash is very important. 

Make the OpenCms home dierectory to your current directory again and
initialize the database with all workplace default files by calling the following command:

\begin{quote}
\texttt{java -mx64M -jar opencmsboot.jar < config/cmssetup.txt}
\end{quote}

Note: The Java binary \texttt{java} should 
reside in the \texttt{bin} folder of your JDK installation. If you have added this folder to your 
system \texttt{PATH} environment as recommended during the installation of the Java VM,
this command should work without giving a path.

If you have a Java runtime environment 
on your system rather than a complete JDK, you
can replace the \texttt{java} command by \texttt{jre}.

If there are errors, check the database connection configuration in your
opencms.properties. Also ensure you are using Java 1.2 or higher since
previous versions are not able to start a JAR file.
the classpath option of your command.

After fixing a problem, you should drop (\texttt{mysqladmin drop opencms42}) and recreate 
the existing database \texttt{opencms42} (see previous section) before 
running the command above for a second time.

After this the OpenCms database setup is completed. You are now able to check the basic OpenCms system. 
Call the OpenCms console by entering:

\begin{quote}
\texttt{java -mx64M -jar opencmsboot.jar}
\end{quote}

This command is similar to the one above.
Note that the part "\texttt{< config/cmssetup.txt}" is missing now.

Log in as Admin and check the accessibility of a workplace file in your database. 
You execute these commands on the shell: 

\begin{quote}
\texttt{login Admin admin\\
readFile "/system/workplace/action/start.html"}
\end{quote}

The OpenCms console answers e.g. with

\begin{quote}
\begin{verbatim}
[Resource]:/system/workplace/action/start.html,
Project=1 , User=2 , Group=3 : Access=rwvr------ : Resource-type=2 : 
 Locked=-1 : lenght=168 : state=0
\end{verbatim}
\end{quote}

\section{Configuring your servlet zone}
Finalize the configuration of your servlet zone. Make sure that your \texttt{jserv.conf} in 
the \texttt{conf} folder of your Apache JServ installation directory contains the line

\begin{quote}
\begin{verbatim}
ApJServMount /servlets /root
\end{verbatim}
\end{quote}

Note: \texttt{ApJServMount}, \texttt{/servlets} and \texttt{/root} are separated by blanks.

The file \texttt{jserv.properties} in the JServ \texttt{servlets} folder must contain

\begin{quote}
\texttt{zones=root\\
wrapper.bin.parameters=-mx64M\\
root.properties=(your servlet configuration path)/zone.properties}
\end{quote}

Note: On Windows systems you should use a backslash instead of a slash. 

Update your \texttt{zone.properties} and set the following repository: 

\begin{quote}
\begin{verbatim}
repositories=(your OpenCms home directory)/opencmsboot.jar
\end{verbatim}
\end{quote}

and add to the \texttt{Servlet Aliases} section the line: 

\begin{quote}
\texttt{servlet.opencms.code=com.opencms.boot.OpenCmsServlet} 
\end{quote}

and to the \texttt{Aliased Servlet Init Parameters} section 

\begin{quote}
\texttt{servlet.opencms.initArgs=opencms.home=(your OpenCms home) }
\end{quote}

After configuring your servlet zone you should restart the Apache to
ensure the new settings are read by the servlet engine.

\section{Now your system is ready}

Start your web server. The OpenCms system is available at\\
http://your.server/servlets/opencms/system/workplace/action/login.html.\\
Login as Admin using the default password admin. 

For any additional information like security or troubleshooting hints
please see the chapter \ref{42install} about the installation of the current OpenCms
major release 4.2 in the OpenCms documentation.
