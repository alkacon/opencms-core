\chapter {Backoffice Modules}
\label{Backoffice Modules}
When writing a module for OpenCms (e.g. a News Module), it is often
required to edit the external data that is used to create the output
with this module. This functionality is called the Backoffice of this
module. A Backoffice can either be an integrated component of a module
or an additional module that will use the functionality and Content
Definition of an already existing module (like the News Module).

\index{Backoffice modules}
Backoffice modules are integrated into the OpenCms administration view,
but it is also possible to add them as a separate view into the system.
Since Backoffice modules are part of the workplace, this chapter will
first show the differences between a {\class "normal"} template class and a
workplace template class and then point out the special needs that are
required to use the predefined Backoffice super class.

\section {Workplace Classes in general}
In principle, workplace template classes are written the same way as
normal template classes, they contain a getContent method that is used
to generate the output of this class which is done with the same three
steps:
\begin{itemize}
\item Read the template file.
\item Set the dynamically generated out
\item Generate the HTML
\end{itemize}

Since the workplace classes need some special extensions to ease their
production, the workplace classes do not extend the CmsXmlTemplate
class, but the CmsWorkplaceDefault class, so that a workplace class
would start like this:
\begin{java}
public class CmsAdministration extends CmsWorkplaceDefault implements
I\_CmsConstants {....}
\end{java}

The {\class CmsWorkplaceDefault} class provides some other additional predefined
user methods that can be used in the templates used for creating the
workplace. Those methods can be used the same way as the default methods
in the normal template classes. The following methods are provided by
the {\class CmsWorkplaceDefault} class (Table~\ref{CmsWPClass}).


\begin{table}
\begin{center}
\begin{tabular}{|l|p{0.70\linewidth}|}
\hline
{\bf Method}&
{\bf Description}\\ \hline
commonPicsUrl()&
This method generates an URL for the common template pics folder.\\ \hline
helpUrl()&
User method to generate an URL for a help file. The
system help file path and the currently selected language will be
considered. The path to the help file folder can be set in the
workplace.ini \\ \hline
picsUrl()&
User method to generate an URL for the system pics folder. \\ \hline
userName()&
User method to get the name of the user.\\ \hline
\end{tabular}
\caption[Methods]{Methods}
\label{CmsWPClass}
\end{center} 
\end{table}


The other difference to the normal template classes is the kind of
template file to be used.  Instead of using a template file of the type
CmsXmlTemplateFile, a special workplace template file named
CmsXmlWpTemplateFile is used in the getContent method.  So a getContent
method for a workplace class would start like this:

\begin{java}
public byte[] getContent(CmsObject cms, String templateFile,\\
\jtabd  String elementName, Hashtable parameters,\\
\jtabd  String templateSelector) throws CmsException \{\\
\jtabe      CmsXmlWpTemplateFile xmlTemplateDocument =\\
\jtabe      new CmsXmlWpTemplateFile(cms,templateFile);\\
....\\
\}\\
\end{java}

This is done because the {\name CmsXmlWpTemplateFile} is able to handle several
additional tags that are not required in normal templates but used to
generate the workplace templates more easier. To use this kind of
template file object, the appropriate XML template must be defined
slighly different than normal. All XML templates so far were defined
this way:

\begin{xml}
<?xml version="1.0" encoding="ISO-8859-1"?>\\
<xmltemplate>\\
\xtaba  <template>\\
\xtaba   ...\\
\xtaba  </template>\\
</xmltemplate>\\
\end{xml}

XML templates to be used with the {\name CmsXmlWpTemplateFile} do not use the
{\tag <xmltemplate>} tag, but the {\tag <workplace>} tag. Besides this difference,
those templates are written the same way than the already known normal
templates

\begin{xml}
<?xml version="1.0" encoding="ISO-8859-1"?>\\
<workplace>\\
\xtaba  <template>\\
...\\
\xtaba  </template>\\
</workplace>\\
\end{xml}

The reason to use a different kind of templates and template objects is
the enhanced functionality that is needed for writing workplace
templates and classes, like multi language support and the additional
tags. For further information of their use, see their usage in the
existing workplace classes and templates. The only tag that is needed
to explained in detail is the <label> tag.

To enable multilanguage support, a workplace template must not contain
any text output coded directly into the template, otherwise it would not
be possible  to change the language for those templates. Therefore, the
{\tag <label>} tag is used to add the required information to the template. An
example for using it is shown in the following code fragment:

\begin{java}
...\\
<tr>\\
<td class="head">]]><LABEL value="label.group" /><![CDATA[</td>\\
</tr>\\
...\\
\end{java}

This tag will be replaced with an corresponding text string depending on
the language used on the workplace. The actual language output is
defined in a language file of the module, see the documentation of the
module mechanism how to do this. In this example, the language
information for this value would stored like this:
\begin{java}
<label><group>Group</group></label>\\
\end{java}
It is highly recommended to use this mechanism to add text output to the
workplace templates and not to add the text directly to the template.

\section {Abstract Backoffice class}
It is now possible to write a workplace template with a corresponding
class that can be used as a Backoffice module for a given module - but
in many cases this would be an inconvenient way. Many module - like a
news module or guestbook -  require a very similar Backoffice
functionality. The single data objects of those models must either be
edited, created or deleted. This can be done with a Backoffice tool
that allows two different views:

\begin{itemize}
\item A list of all available data objects which can be configured
with various display filters. Data objects can be deleted in this view.
       
\item An input form to edit existing or adding new data objects.
\end{itemize}

To speed up the development of new Backoffice modules, OpenCms provides
a predefined abstract class (and templates) that enables the
functionality of the data list.  It contains a {\meth getContent} method to
produce the output of the data list template based on the information
provided by a Content Definition. This class can be easily extended and
modified for each individual Backoffice module. Only the input form -
which is always depending on the data fields of a module - must be
implemented separately. The list of all available data objects that is
generated with the predefined classes looks like figure~\ref{AbstactBO}.

\begin{figure}
\begin{center}
\includegraphics[clip,width=\sgw]{pics/backOffice/list}
\end{center}
\caption[List view in the Abstract Backoffice]
    {List view in the Abstract Backoffice}
\label{AbstactBO}\jtabd
\end{figure}

To use the abstract class, it has to be extended by an implementation
class that extends several configuration methods, providing the
necessary data about the Content Definition to be used and the URLs of
the forms required for adding, editing or deleting data. The following
methods have to be overwritten in the implementation class to archive
this:
\begin{itemize}
\item getContentDefinitionClass(): Returns the Content Definition
class used by this implementation
\item getCreateUrl: Returns the URL of the dialog to add a new data
entry.
\item getEditUrl: Returns the URL of the dialog to edit an existing
data entry.
\item getDeleteUrl: Returns the URL of the dialog to delete an
existing data entry.
\item getSetupUrl: Return the URL of the setup dialog
\end{itemize}

The returned URLs of those methods direct to the dialogs (i.e. templates
and classes) that have to be produced in the process of creating the
Backoffice Module. Those dialogs have to be adapted to the requirements
of the data fields of the Content Definition which should be possible to
be edited. When those dialogs are called, they require a special
information, which data object should be editied or deleted.
This is done by adding the URL-Parameter {\code ?id=value} to the URL. The value
given is either the unique id of the data object (which can be accessed
by calling  the {\meth getUniqueId()} method of the used Content Definition) or
the keyword "new" to add a new data object.
In most cases the {\meth getCreateUrl()} and {\meth getEditUrl()} will return the same
value, i.e. the same dialog will be shown when adding a new data entry
or editing an existing one, the second option would require that the
input fields of the form would be preset with the existing values.

A sample implementation of the News Backoffice has the following
implementation of those abstract classes:

\begin{java}
public class NewsBackoffice extends A\_CmsBackoffice \{\\
\jtaba        public Class getContentDefinitionClass() \{\\
\jtabb                  return NewsContentDefinition.class;
\jtaba        \}\\
\jtaba        public String getCreateUrl(CmsObject cms) \{\\
\jtabb                return "NewsEditBackoffice.html";\\
\jtaba        \}\\
\jtaba        public String getDeleteUrl(CmsObject cms) \{\\
\jtabb                return "NewsEditBackoffice.html";\\
\jtaba        \}\\
\jtaba        public String getEditUrl(CmsObject cms) \{\\
\jtabb                return "NewsEditBackoffice.html";\\
\jtaba        \}\\
\}\\
\end{java}
\index{getContentDefinitionClass()}
\index{getCreateUrl(CmsObject cms)}
\index{getDeleteUrl(CmsObject cms)}
\index{getEditUrl(CmsObject cms)}

Using the predefined classes, the list of data objects of a given
content definition can be created without writing new template files.
They are required for the create, edit and delete dialogs. Since those
dialogs depend on the data to be modified, there is no general rule how
to write the templates and the belonging classes. Those dialogs are made
the same way than a normal form in the OpenCms (see the chapters about
creating forms in OpenCms). There are some formalities to comply with
when writing those forms:
\begin{itemize}
\item The form class must extend the class CmsWorkplaceDefault.
\item The template file object must be an instance of
CmsXmlWpTemplateFile.
\item The form template must be of the XML type <workplace>.
\item The information, which data object has to be edited is given as
the URL-Parameter ?id=value .
\item The form class must use the given Content Definition to access
the data object.
\end{itemize}

It is important that the template classes for those dialogs are able to
check the entered data for correct inputs before storing the values in
the Content Definition. For example, it should be checked  if a given
date is correct or an email address is formed after the correct syntax.

The dialog to edit an existing news entry could look like figure~\ref{AbsrtactBO1}.

\begin{figure}
\begin{center}
\includegraphics[clip,width=\sgw]{pics/backOffice/new}
\end{center}
\caption[Abstract Backoffice class Pic 2]{Abstract Backoffice class Pic 2.}
\label{AbsrtactBO1}
\end{figure}

The Java class to process this dialog has to do the following steps:
First to read all given parameters from the input fields, then to check
if all data values are non empty,  to create a new Content Definition
object with the given data and finally to store it in the data storage.

%====================================================================
\section{Changes to the Abstract Backoffice class since OpenCms 4.4}
OpenCms uses the abstract backoffice class {\name A\_CmsBackOffice} from the
package  \\
{\name com.opencms.defaults} to make the writing of new backoffice 
modules more easy. 
This part describes the changes to the existing mechanism and how these
changes can be used.

\subsection{Changes to the getContent... methods}
The abstract backoffice class defines three methods for processing the 
input-form of the backoffice:

\begin{itemize}
\item getContentEdit: form to modify an existing dataset
\item getContentNew:  form to create a new dataset
\item getContentDelete: form to delete a dataset
\end{itemize}

The parameters of these getContent... methods are now the following:

\begin{java}
public {\bf String} getContentEdit(CmsObject cms,\\
\jtabf CmsXmlWpTemplateFile templateFile,\\
\jtabf {\bf A\_CmsContentDefinition cd},\\
\jtabf String elementName,\\
\jtabf {\bf Enumeration keys},\\
\jtabf Hashtable parameters,\\
\jtabf String templateSelector)\\
\jtabf throws CmsException
\end{java}
\index{Enumeration keys}
The modifications to the existing version are highlighted.

\begin{itemize}
\item A\_CmsContentDefinition cd: gives the correct instance of the Content Definition
    to the method. This method is either a new (empty) CD or a CD which includes
    an existing dataset.This way you can always access the correct CD, without 
    fetching it with your own programm code.
\item Enumeration keys: Inherits the names of all data which comes from the form
    (URL and form-paramter). It can be used to determine which parameters had 
    been on a certain page, when working with forms over multiple pages.
\item return-value String: The method returns a String. An empty String signals that
    no error has occurred while processing the form. If an error has occurred, the 
    String contains the error-message which will be displayed in the template.
\end{itemize}

\subsection{Additional method getSetupUrl}
Now it is possible to add an additional setup page to the module, which can be
called from the backoffice. This does the method

\begin{java}
public String getSetupUrl(CmsObject cms, String tagcontent, A\_CmsXmlContent doc, 
Object userObject) throws Exception
\end{java}
\index{getSetupUrl() }

This method has to be overwritten from your own class, which extends the class
{\name A\_Cms\-Backoffice}, if this function is wanted. The method returns the URL 
of the setup page of the module. This page can be put at any place within the module
directory.
\index{setup page}

If a setup page is defined in this way, a button to access this page is shown in 
the list view of the module.

\subsection{Filling the template from the CD automatically}
The main target of the changes of abstract backoffice class is to decrease the
sumpturay of creating an own backoffice class. In order to let the abstract backoffice
do as many tasks as possible, some conventions concerning names of variables, methods,
datablocks and fields have to be known.
Table~\ref{nameConv} shows an example for these naming conventions.

\begin{table}
\begin{center}
\begin{tabular}{|c|c|c|c|}
\hline
member &
get-method&
name of&  
name of \\
variable&
&
datablock&
inputfield\\ \hline  
m\_text&
public String&
<process>text</process>&
text\\
&
getText()&
&  \\ \hline
\end{tabular}
\caption[name conventions]{name conventions}
\label{nameConv}
\end{center} 
\end{table}

With the help of this naming scheme, the form is automatically filled with the right
values.

The methods {\name getXYZ} have to return a String to enable the automatically assignment
of the data to the fields. In the case that the method {\name getXYZ} accesses a value
that is no String, another method {\name getXYZString} has to be written, which returns 
the value as a String.
The returned value is then inserted into the datablock {\name XYZ}.

Member variables have always to be initialized, with 'null' in case of an Object,
and '-1' in case of an integer.

\subsection{Filling the CD from the template's fields automatically}
Just like filling the datablocks before, the assignment of the data from the 
templateform to the CD takes place automatically. The naming scheme for this 
is the same as before (Table~\ref{nameConv2}).

\begin{table}
\begin{center}
\begin{tabular}{|c|c|c|c|}
\hline
member &
set-method&
name of&  
name of \\
variable&
&
datablock&
inputfield\\ \hline  
m\_text&
public String&
<process>text</process>&
text\\
&
setText()&
&  \\ \hline
\end{tabular}
\caption[name conventions]{name conventions}
\label{nameConv2}
\end{center} 
\end{table}

\subsubsection{special case: file upload}
To make it possible to read out uploaded files automatically, some further
naming conventions has to be followed (Table~\ref{nameConv3}).
\index{file upload}

\begin{table}
\begin{center}
\begin{tabular}{|c|c|c|c|}
\hline
member &
set-method&
name of&  
name of \\
variable&
&
datablock&
inputfield\\ \hline  
m\_upload&
public void&
<process>text&
Upload\\
&
setUpload(String filename)&
</process>&  \\ \hline
m\_uploadcontent&
public void&
---&
---\\
&
setUploadContent(byte content)&
& \\ \hline

\end{tabular}
\caption[name conventions]{name conventions}
\label{nameConv3}
\end{center} 
\end{table}

This means that two values are always stored, the name of the file which was
uploaded as well as the content of the file.
The name of the inputfield in the template is used as a key for the name within 
the Content Definition. The according method name to set the content is 
derived from that.

\subsection{Testing for errors}
A method in the Content Definition tests the values which have been inserted.
This method is called automatically before the data is stored to the database.

\begin{java}
public void check(boolean finalcheck) throws CmsPlausibilizationException
\end{java}
\index{check() }
\index{CmsPlausibilizationException}

All data has to be tested within this method and a CmsPlausibilizationException
has to be thrown in case of an error. The exception contains a vector of 
error codes, which is passed to the constructor of the exception.

\begin{java}
public CmsPlausibilizationException(Vector error)
\end{java}

The certain errorcodes within the vector are Strings which are build 
after the following scheme:

'fieldname\_error'

'Fieldname' is the inputfield in which the error has occurred and error is
the kind of error (e.g. 'empty'). Errormessages according to these errorcodes
are searched within the backoffice. There is a mechanism with 3 stages, which
serves as a naming scheme for the datablocks inside the template. The following
datablocks are searched one after another and issued in case one was found.
This way it is possible to have different graduations of the error messages.

\begin{itemize}
\item <errfieldname\_error>errorcode<errfieldname\_error>: This datablock 
    defines an error text which is issued when the error code
    'fieldname\_error' appears.
\item <errfieldname>errorcode<errfieldname> This datablock defines
    an error for the inputfield 'fieldname'. It is used if the 
    datablock \\
    '<errfieldname\_error>errorcode<errfieldname\_error>' could
    not be found.
\item <errerror>errorcode<errerror>: This datablock defines an error text for
    a kind of error. It is used if the datablock 
    '<errfieldname>errorcode<errfieldname>' could not be found.
\end{itemize}

If no error text can be found for the error code, the error code is given 
directly to the template.




%%% Local Variables:
%%% mode: latex
%%% TeX-master: "OpenCmsDoc"
%%% End:
